%%%%%%%%%%%%%%%%%%%%%%%%%%%%%%%%%%%%%%%%%%%%%%%%%%%%%%%%%%%%%%%%%%%%%%%%%%%%%
%
%                     Copyright 2010 Jim Finnis.
%                           All Rights Reserved
%
%
%  System        : 
%  Module        : 
%  Object Name   : $RCSfile$
%  Revision      : $Revision$
%  Date          : $Date$
%  Author        : $Author$
%  Created By    : Jim Finnis
%  Created       : Sun Jan 3 15:35:05 2010
%  Last Modified : <100103.1543>
%
%  Description 
%
%  Notes
%
%  History
% 
%%%%%%%%%%%%%%%%%%%%%%%%%%%%%%%%%%%%%%%%%%%%%%%%%%%%%%%%%%%%%%%%%%%%%%%%%%%%%
%
% Copyright (c) 2010 Jim Finnis.
% 
% All Rights Reserved.
% 
% This  document  may  not, in  whole  or in  part, be  copied,  photocopied,
% reproduced,  translated,  or  reduced to any  electronic  medium or machine
% readable form without prior written consent from Jim Finnis.
%
%%%%%%%%%%%%%%%%%%%%%%%%%%%%%%%%%%%%%%%%%%%%%%%%%%%%%%%%%%%%%%%%%%%%%%%%%%%%%

\chapter{Introduction}

SCMS was designed as a system for creating simple websites of a few, mostly
static pages, although it can be used for larger systems. Most large CMSs
use an SQL database to store their data, but the main rationale for SCMS
is that it requires no such database --- all the data is contained in files.
This means that it can be developed and maintained on a test server, rather
than on the live site, with files being transferred to the live site using
a revision control system such as Subversion.

SCMS is a self-contained PHP system, requiring no other installable modules.
Like most CMSs, SCMS uses a template system so that many pages can be made
up of the same HTML and CSS elements, with only the actual content changing.
In addition, SCMS handles multilingual sites elegantly: there is a default
set of data for each template and page which can be overriden by data in 
a language subdirectory.

