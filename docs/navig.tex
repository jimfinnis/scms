%%%%%%%%%%%%%%%%%%%%%%%%%%%%%%%%%%%%%%%%%%%%%%%%%%%%%%%%%%%%%%%%%%%%%%%%%%%%%
%
%                     Copyright 2010 Jim Finnis.
%                           All Rights Reserved
%
%
%  System        : 
%  Module        : 
%  Object Name   : $RCSfile$
%  Revision      : $Revision$
%  Date          : $Date$
%  Author        : $Author$
%  Created By    : Jim Finnis
%  Created       : Sun Jan 3 15:35:58 2010
%  Last Modified : <110116.1341>
%
%  Description 
%
%  Notes
%
%  History
% 
%%%%%%%%%%%%%%%%%%%%%%%%%%%%%%%%%%%%%%%%%%%%%%%%%%%%%%%%%%%%%%%%%%%%%%%%%%%%%
%
% Copyright (c) 2010 Jim Finnis.
% 
% All Rights Reserved.
% 
% This  document  may  not, in  whole  or in  part, be  copied,  photocopied,
% reproduced,  translated,  or  reduced to any  electronic  medium or machine
% readable form without prior written consent from Jim Finnis.
%
%%%%%%%%%%%%%%%%%%%%%%%%%%%%%%%%%%%%%%%%%%%%%%%%%%%%%%%%%%%%%%%%%%%%%%%%%%%%%

\chapter{Navigation}
\label{navig}
In this chapter we'll discuss navigation menus in more detail, describing the
different types of navigation item which can be created and how more complex
menus can be built. It's worth going into this in some detail because menu
generation is probably the most complex part of writing tag definition files.

\section{The navigation file}
\useglosentry{navfile}
\useglosentry{navtree}
As described in Chapter \ref{trees}: Trees and Menus, the navigation menu is a
tree collection created automatically from the file \texttt{site/navigation}.
This file is a text file which you create, which describes the page
organisation of your site\footnote{\emph{not} the directory organisation, as I
may have mentioned a few times before.}. Here's an example:
\begin{MyVerbatim}
others/home
products
-chairs
--chairs/chair1
--chairs/chair2
--chairs/chair3
-tables
--tables/table1
--tables/table2
--tables/table3
-desks
--tables/desks/desk1
--tables/desks/desk2
--tables/desks/desk3
--tables/desks/desk4
-recentprods
--tables/table2
--tables/desks/desk3
--tables/desks/desk4
--chairs/chair2
others/contact
others/help
\end{MyVerbatim}
Each entry is the name of a page (i.e. a file in the \texttt{site/pages}
directory.) This will give us a home page, a products page, a contact page, and
a help page all at the top level. Note that some of the page data is kept in
subdirectories - the contact and help page files are kept in the
\texttt{others} subdirectory although they appear at the top level in the menu
hierarchy. This navigation file will automatically generate the
\texttt{navtree} tag's tree collection, which we can iterate over to render
the menus as described in Chapter \ref{trees}: Trees and Menus.

Note that the products menu contains four submenus. Three are for different kinds of products
and one is for recent products. This recent products menu contains items which are already in
some of the other menus: \emph{pages can be referenced in a navigation menu in more than one place.}

\subsection{Directory items}
In the above hierarchy, the chairs, tables and desks entries are actually
pages in their own right. However, we can see that those entries refer to
directories (since they contain other files for the actual product pages.) To
make this work, we have to create a \texttt{defaults} file in each of these
directories giving the page content when the directory is specified, which
contains a line giving a value for the \texttt{standalone} tag. For example:
\indtagsec{page:standalone}
\begin{MyVerbatim}
name=Tables
standalone=yes
content=[[
    <p>This page describes the various tables we have available.</p>
]]
\end{MyVerbatim}
As described in Chapter \ref{chappages}: Pages, this allows a directory to be
a page. If one of the subpages (such as \texttt{tables/table1}) is loaded, the
\texttt{table/defaults} tag will still be loaded but its tags will be
overriden by those in the page file.
If we don't create \texttt{defaults} files with \texttt{standalone} tags in
all three items which are directories, the result will be a 404 when we try to
open them.

\section{Heading-only items}
Another alternative is to stop those items from being pages at all --- just
have them as section headings in the menu. This requires two changes: marking
the items as heading-only, and rewriting the menu tags to process that
information. We do the first part by simply putting square brackets around the
heading-only items. The \texttt{site/navigation} file therefore becomes:

\begin{MyVerbatim}
others/home
products
-[chairs]
--chairs/chair1
--chairs/chair2
--chairs/chair3
-[tables]
--tables/table1
--tables/table2
--tables/table3
-[desks]
--tables/desks/desk1
--tables/desks/desk2
--tables/desks/desk3
--tables/desks/desk4
others/contact
others/help
\end{MyVerbatim}
\indtagsec{navigation tree node fields!type}
What this does is set the \texttt{type} field of the item in the menu tree to
\texttt{H} rather than the normal \texttt{N}. We now need to modify the menu
code to stop rendering links for these items.
We'll assume we're writing these tags in a \texttt{.tags} file in the template's
directory, so all tags we define are prefixed with \texttt{template:}. First,
we'll define a couple of tags which will render an item in the menu with or
without a link:
\begin{MyVerbatim}
## don't show a link
nolinknode={{a:name}}

## show a link
linknode = <a href="{{a:url}}">{{a:name}}</a>
\end{MyVerbatim}
This is pretty straightforward. The \texttt{nolinknode} tag will render the
current tree node without a link, as just a list item; while the
\texttt{linknode} tag will render it with a link. These two tags are stored
directly into the template system without being expanded until they're used,
\indtagsec{rendertree}
which will be from inside a \texttt{rendertree} tag, so the \texttt{a:...}
items will be in the right context. Now we add a third tag, which will show
the node as a link only if the node's \texttt{type} is not \texttt{H} -- i.e.,
it's not a heading. We do this using the \texttt{streq} tag, which takes four
\indtag{streq}
\label{strequselab}
arguments --- the two strings to compare, the string to use if the strings are
equal, and the string to use if they are not equal:
\begin{MyVerbatim}
## show a link unless this is a heading-only node

linknodeifnotheading=[[
        {{streq|{{a:type}}|H|
            {{template:nolinknode}}|
            {{template:linknode}}
        }}
]]
\end{MyVerbatim}
Now we're ready to set up the menu using the menu tags, and render it. This we
do from inside a tag definition, since that's the only place we can use tags.
The tag we'll define is \texttt{navmenu}, and it'll set up the tree rendering
environment and then render it:
\begin{MyVerbatim}
navmenu=[[      
    ## first define the prefix and suffix: we'll render
    ## the menu as an unordered list, and each level within
    ## it as another list. Each level, therefore, gets
    ## wrapped in <ul></ul>

    {{treeprefix|default|<ul>}}
    {{treesuffix|default|</ul>}}
\end{MyVerbatim}
\indtagsec{treeunselnode}
The next step is to use the \texttt{treeunselnode} tag to tell SCMS how to
render an unselected tree node. In this case, we call the
\texttt{linknodeifnotheading} tag we defined before to render a link if the
item isn't a heading:
\begin{MyVerbatim}
    ## unselected nodes have a link unless they're
    ## heading only
    
    {{treeunselnode|default|<li>
        {{template:linknodeifnotheading}}|</li>}}
\end{MyVerbatim}
\indtagsec{treeselnode}
Now we use \texttt{treeselnode} to set up how to render a selected node.
This will just render the node without a link, since we don't need a link
when we're already viewing this page:
\begin{MyVerbatim}
    ## selected node renders with no link
    {{treeselnode|default|<li>
            {{template:nolinkitem}}|</li>}}
\end{MyVerbatim}
\useglosentry{trail}
Finally, we need to render items in the trail --- i.e. on the path
down to the selected item --- in the same way as unselected nodes. This
\indtagsec{treetrailnode}
is easy: we just don't specify the \texttt{treetrailnode} template,
and it will fall back to using the template specified by the \texttt{treeunselnode}
tag as described in section~\ref{unspectreetemps}.

All we need to do now is mark the tree, so it knows which items are selected,
and render it:
\begin{MyVerbatim}
    {{marktree|{{navtree}}|spec|{{spec}}}}
    {{rendertree|{{navtree}}|a}}
]]
\end{MyVerbatim}

Of course, there's nothing to stop us writing the whole thing without defining
the extra templates:
\begin{MyVerbatim}[commandchars=+\[\]]
    {{treeprefix|default|<ul>}}
    {{treesuffix|default|</ul>}}

    {{treeunselnode|default|<li>
        {{streq|{{a:type}}|H|
            +textit[{{a:name}}]|
            <a href="{{a:url}}">{{a:name}}</a>
        </li>}}
    }}
    {{treeselnode|default|<li>
            +textit[{{a:name}}]|</li>}}

    {{marktree|{{navtree}}|spec|{{spec}}}}
    {{rendertree|{{navtree}}|a}}
\end{MyVerbatim}
and that's perfectly clear (or should be), but note that we're duplicating the
`render without a link' template: I've marked it in \textit{italics} in the
code above. It appears as the selected node template, and also in the
unselected or trail node template as the `do if true' part of the `is this a
heading only?' condition. If we wanted to change it later, we'd have to change
it in both places.

\clearpage
\section{Invisible items}
\label{invisitems}
In a similar way to heading-only items, items can be marked as `invisible.'
This is done by putting them in round brackets in the navigation file:
\begin{MyVerbatim}
others/home
products
-[tables]
--tables/table1
--tables/table2
--tables/table3
-[desks]
--tables/desks/desk1
--tables/desks/desk2
--tables/desks/desk3
--tables/desks/desk4
(adminlogin)
others/contact
others/help
\end{MyVerbatim}
would add an \texttt{adminlogin} page which should not be shown by the menu
system. As with heading-only pages, the system does nothing with these except
change the \texttt{type} field in the navigation tree nodes, in this case to
\texttt{I}. It's up to you to program your menus to implement this. Here's
an example extending the previous example and using a \texttt{switch} tag. Actually,
we've used two such tags, one for the each part of the node's template. We don't
need to change the selected node templates, because a selected node can't be invisible
or heading-only:

\begin{MyVerbatim}
navmenu=[[
    {{treeprefix|default|<ul>}}
    {{treesuffix|default|</ul>}}
    
    {{treeunselnode|default|
        {{switch|{{a:type}}|
            H|<li>{{a:name}}|
            I||
            default|<li>{{template:link}}{{a:name}}</a>}}|
        {{switch|{{a:type}}|
            H|</li>|
            I||
            default|</li>}}
    }}        
    {{treeselnode|default|
            {{streq|{{a:type}}|I||
                <li>{{a:name}}|</li>}}}}

    {{marktree|{{navtree}}|spec|{{spec}}}}
    {{rendertree|{{navtree}}|a}}
\end{MyVerbatim}
\indtagsec{switch}
The \texttt{switch} tag takes an input to compare, and then any number of string/output pairs:
\begin{MyVerbatim}
{{switch|input|str1|output1|str2|output2...|default|defaultoutput}}
\end{MyVerbatim}
If the input matches a string, the corresponding output is produced. If no string is matched by
the time we get to \texttt{default}, the default output is produced. If there's no default item
we return the empty string.

So in our code above, the unselected node template works by using a three-way switch:
if the type is \texttt{H}, it's a heading and we output just the name with no link; if the type
is \texttt{I}, it's invisible and we output nothing; otherwise we output the link.
\indtagsec{streq}
The selected node works using the \texttt{streq} tag, outputting nothing if the link is invisible
and just the name otherwise.
\clearpage
\section{Menus with different levels}
In this section we'll discuss common styles of menu which require different
levels of the menu to be rendered in different ways, and show some examples of
how this can be done.

\subsection{Simple menu with different levels}
This is the simplest, and perhaps most common case: rendering the menu as different classes
of HTML unordered list at different levels. We can do this very simply by using different tree prefixes
for the different levels:
\begin{MyVerbatim}
navmenu=[[
{{treeprefix|0|<ul class=level1>}}
{{treeprefix|1|<ul class=level2>}}
{{treeprefix|2|<ul class=level3>}}
{{treesuffix|default|</ul>}}

{{treeunselnode|default|<li>
{{switch|{{a:type}}|
    H|{{a:name}}|
    I||
    default|<a href="{{a:url}}">{{a:name}}</a>
    }}|
    </li>
}}

{{treeselnode|default|<li><b>{{a:name}}</b>|</li>}}

{{marktree|{{navtree}}|spec|{{spec}}}}
{{rendertree|{{navtree}}|a}}
]]
\end{MyVerbatim}
Note that I've used \texttt{switch} to deal with invisible and heading-only nodes, and that
trail nodes are the same as unselected nodes. I've only used one switch in that first case,
so invisible nodes will show up as empty list items! I leave fixing it as an exercise.

\subsection{Single level menu at all levels}
\label{simplesingle}
In this system, we just show a single level of the menu, and the name of the
parent node to allow us to move back up the hierarchy. For example, if we were
on the home page, our example system would show something like
\begin{center}
\begin{tabular}{llll}
\texttt{Home} & \texttt{Products} & \texttt{Contact} & \texttt{Help} \\ 
\end{tabular}
\end{center}
If we clicked on `Products', we'd get
\begin{center}
\begin{tabular}{llll}
\texttt{\emph{Home}} & \texttt{Chairs} & \texttt{Tables} & \texttt{Desks} \\
\end{tabular}
\end{center}
Clicking on `Chairs', we'd get 
\begin{center}
\begin{tabular}{llll}
\texttt{\emph{Products}} & \texttt{chair1} & \texttt{chair2} & \texttt{chair3} \\
\end{tabular}
\end{center}
and finally clicking on `chair1' would give us
\begin{center}
\begin{tabular}{l}
\texttt{\emph{Chairs}}
\end{tabular}
\end{center}
the only link available being one to return us to the previous level. In other
words, we show the name of the parent node and the children of the current
node. Instead of using tree rendering, we use the tags from section~\ref{treenav}, Navigating the Tree.
\begin{MyVerbatim}
navmenu=[[
    {{ifnotempty|{{parent|{{curnavnode}}}}|
        {{withnode|{{parent|{{curnavnode}}}}|a|
            <i><a href="{{a:url}}">{{a:name}}</a></i>}}
        |}}
    {{withnode|{{curnavnode}}|a|
        {{iftagexists|a:child|
            {{foreach|{{a:child}}|b|
                <a href="{{b:url}}">{{b:name}}</a>,
            }}
        |}}
    }}
]]
\end{MyVerbatim}
\indtag{curnavnode}
This uses \texttt{curnavnode}, which contains a node handle to the node in the navigation tree
corresponding to the current page. Then get the node handle of this node's parent, and if it's
not empty (i.e. this is not the root node) we use \usetag{withnode} to render it\footnote{I'm using
the \texttt{<i>} tag here because it's easy, not because I think it's a good idea. You should, of course,
use a \texttt{<div>} or \texttt{<span>}.}.
We then use another \usetag{withnode} to access the current page's node itself. Inside this, we check to see if the 
node has a child. If it does, we iterate over that child, outputting the \texttt{url} and \texttt{name}
tags of the subpages.

\subsection{Breadcrumbs}
\useglosentry{breadcrumbs}
A breadcrumb trail is a list of links to all the ancestor pages of the current page. We can create
one by rendering the navigation tree with empty templates for unselected pages, so we only see
the selected (current) page and those pages in the trail:
\begin{MyVerbatim}
breadcrumbs=[[
{{treeprefix|default|}}
{{treesuffix|default|}}
{{treeunselnode|default||}}
{{treeselnode|default|{{a:name}}|}}
{{treetrailnode|default|<a href="{{a:url}}">{{a:name}}</a> |}}
{{marktree|{{navtree}}|spec|{{spec}}}}
{{rendertree|{{navtree}}|a}}
]]
\end{MyVerbatim}
We've specified that the trail pages are rendered by links, while the selected pages doesn't have
a link --- after all, you're already viewing that page.

\subsection{A complex example using \texttt{loadpage} to read data from subpages}
\label{loadpageexample}
This is a menu where the two levels are utterly different, and rendered using
separate code. The system I'll describe here is one we use on the Broadsword
website, where a single level main menu appears at the top of all pages, but
there is a special `apps' menu which appears as part of the \texttt{apps} page
content in which the main menu is suppressed, and information about each
subpage of \texttt{apps} is shown by reading the subpage data itself.

Here's the \texttt{navigation} file we're using:
\begin{MyVerbatim}
home
services
news
apps
-a/myfirstapp
-a/anotherapp
contact
support
\end{MyVerbatim}
We can see that there is mainly one level, except for the children of \texttt{apps}. That
page is dealt with specially, as we mentioned above.

The main menu is essentially straightforward, using \usetag{foreach} to iterate through the current
level of the navigation tree. However, the requirements of the page layout mean there are
a few little complexities. I'll deal with the code bit by bit.
\begin{MyVerbatim}
navmenuwidth=450
navmenuleft={{sub|512|{{div|{{template:navmenuwidth}}|2}}}}
\end{MyVerbatim}
This calculates the dimensions of the menu div for setting in CSS. We
set the menu div to be 450 pixels wide, and then given that the parent
div in which it will appear is 1024 pixels wide, the left hand edge of
the menu div can be set to $512-\frac{navmenuwidth}{2}$. With this information we 
can now code the \texttt{template:navmenu} tag itself.
\begin{MyVerbatim}
navmenu=[[
    {{set|curcoll|{{findcollection|{{navtree}}|spec|{spec}}}}}
\end{MyVerbatim}
Here we're starting the menu tag, and storing the current level in a variable
called \texttt{curcoll}. This is a slightly different approach from that used
in Section~\ref{simplesingle}, using \usetag{findcollection} to find the the current node's
\emph{collection} rather than \usetag{findnode} to find the node itself.
\begin{MyVerbatim}
{{iftagexists|page:nomenu||
    {{marktree|{{navtree}}|spec|{{spec}}}}
\end{MyVerbatim}
First we check to see if the current page has a tag called \texttt{nomenu}. If so, 
we don't process the rest of the tag --- you'll see from the rest of the code
that the rest of the tag is inside the false part of the \usetag{iftagexists} condition
and so only runs if there's no \texttt{nomenu}. Note that double bar: the true part
of the condition is the empty string.
\begin{MyVerbatim}
    <div id="navmenu" style="width:{{template:navmenuwidth}}px;
                            left:{{template:navmenuleft}}px;">
\end{MyVerbatim}
This starts the navigation menu div, setting the position and width to the values we
worked out earlier.
\begin{MyVerbatim}
    {{foreach|{{curcoll}}|a|
        {{streq|{{a:type}}|I||
\end{MyVerbatim}
Now we use \usetag{foreach} to iterate through each item in the current menu level, only showing those for
which the \texttt{type} field is not \texttt{I} -- i.e. which are not hidden.
\begin{MyVerbatim}
        {{iftagtrue|a:issel|
            <span>{{a:name}}</span>|
\end{MyVerbatim}
\indtagsec{tree item fields!issel}
If the node's \texttt{issel} field is true -- i.e. it is selected -- we just render the name of the page
as a span. This is picked up by the CSS and rendered accordingly, with no link.
\begin{MyVerbatim}
            <a href="{{a:url}}"
                {{streq|{{a:name}}|apps|
                    onmouseout="P2H_StartClock();"
                    onmouseover="P2H_Menu('PMgamesbutton',0,0);"
                |}}>
            {{a:name}}</a>}}
\end{MyVerbatim}
However, if the item is not selected, we render it as a link. There's another complication here - if
the name of the current node is \texttt{apps} we modify the link to provide scripted events. The idea
is that stuff appears while we're hovering over the apps link.
\begin{MyVerbatim}
        }}
    }}
    </div>
}}

]]
\end{MyVerbatim}
Finally, we close off the tags, the div and the definition of \texttt{navmenu}.

\subsubsection{Reading in data from other pages}
Now for the second level of the menu, which appears in the \texttt{apps}
page file. This is rather clever, in that it will not only display information which
the navigation menu knows about the pages, but it will also open each page and read
the tags it defines so we can show some additional information about each app.
Consider that each subpage of \texttt{apps} looks something like this:
\begin{MyVerbatim}
name=My Application
shorttext=A short description to appear in the list of apps
imgcode=myimage.png
content=[[<p>A much longer text describing the content
of the app's page in full, which we'll only see
when we view the page itself.</p>]]
\end{MyVerbatim}
We will be able to use some of the tags defined by the subpages inside
the \texttt{apps} page which lists them.

First, we're going to define the content of the page:
\begin{MyVerbatim}
content=[[
{{ifnotempty|{{curnavnode|child}}|
\end{MyVerbatim}
This uses \usetag{curnavnode}, which we know always contains a pointer to the current node in the navigation tree.
\indtagsec{ifnotempty}
Here, we only run the following code if the \texttt{child} field in \texttt{curnavnode} is
a non-empty string: i.e., if the current page has child pages. This uses
the syntax in Section~\ref{usenodetag}.
\begin{MyVerbatim}
    <div id="listcontainer">
        {{count|i|start}}
\end{MyVerbatim}
Now we have created a list div and inside it, set up a \emph{counter} using
the \usetag{count} tag. This is a value which
can be incremented, and used for various purposes. We're going to use it to set up `zebra
striping' so alternate entries in the list are different styles.
\useglosentry{zebra}
\begin{MyVerbatim}
        {{foreach|{{curnavnode|child}}|a|
            {{loadpage|{{a:spec}}|app}}
            {{page:appentry}}
        }}
    </div>
|}}        
]]
\end{MyVerbatim}
Now we can start iterating through the collection of the current node's children. In this case
these are the pages which are subpages of the \texttt{app} page. For each page, we call the
\usetag{loadpage} tag. This is a tag which will read the page file whose specifier is given
in the first argument, but instead of prefixing each tag defined therein with \texttt{page:}, it
will use the prefix given in the second argument. For example, \texttt{app:name} will be defined
after the tag has run, as will all the other tags in the page loaded. Once the subpage we're currently
iterating is loaded, we expand the tag \texttt{page:appentry}, which we'll define shortly.
\clearpage
So to recap, this code:
\begin{itemize}
\item Finds the current page's node
\item Checks that it has children, and if so...
\item Starts a counter called `i'
\item For each child of the current page
\begin{itemize}
\item Load that page's page file, and define all its tags but prefixed with `app' instead of `page'
\item Expand \texttt{page:appentry} which will use the information we read out of the subpage's file
to display some information about the subpage.
\end{itemize}
\end{itemize}
Now let's look at that \texttt{page:appentry} tag:
\begin{MyVerbatim}
appentry=[[
    <div class="{{count|i|alt|listitemeven|listitemodd}}"
        style="background-image:url({{imgroot}}{{app:imgcode}})">
        <h1><a href="{{a:url}}">{{app:name}}</a></h1>
        {{app:shorttext}}
    </div>
]]
\end{MyVerbatim}
The first line creates a class, and demonstrates how the \usetag{count} tag works: we created
the counter `i' at the start of the page, and each time round we read and increment it in
\texttt{alt} mode. In this mode, the tag will return either its third or fourth item, depending
on whether the counter is even or odd. This means that we'll get divs in alternate classes.
If we set the background colour slightly differently in each one, we'll see the `zebra striping'
effect we want. The remaining lines use various tags defined inside the subpage's file and
loaded in using \usetag{loadpage}: \texttt{app:name} is the page's name, \texttt{app:shorttext}
is a tag we add to each app's page to give a brief description for use in this list, and
\texttt{app:imgcode} is the name of an image file in the image root directory (set to be \texttt{site/images/},
the value of \usetag{imgroot}).
\clearpage
\subsection{Two-level horizontal menu}
This style of menu is used on sites like \emph{The Register}. It features two horizontal
menus --- the top level menu, and the current submenu. For example, consider the navigation file
\begin{MyVerbatim}
(home)
-h/alerts
-h/newsletters
-h/reviews
hardware
-hw/pcs
-hw/servers
-hw/hpc
software
-sw/os
-sw/apps
-sw/dev
\end{MyVerbatim}
which is a truncated version of the Register's structure. There would also be additional pages
below each sublevel for the individial news stories The home page is special, in that there
is no menu entry for it: you can always get there by clicking on the logo. So on the home page,
the menus that appear are:
\begin{MyVerbatim}[commandchars=+\[\]]
top line:    hardware   software
bottom line: alerts newsletters reviews
\end{MyVerbatim}
In Software we get:
\begin{MyVerbatim}[commandchars=+\[\]]
top line:    hardware   +emph[software]
bottom line: os apps dev
\end{MyVerbatim}
If we go into apps or any page below it, we get:
\begin{MyVerbatim}[commandchars=+\[\]]
top line:    hardware   +emph[software]
bottom line: os +emph[apps] dev
\end{MyVerbatim}
Clearly the highlighting can be done by using the existing trail mechanism. Let's look at an
implementation. First, we'll mark the tree as usual, and then render the root menu directly using
a \texttt{foreach}:
\begin{MyVerbatim}
navmenu=[[
    <div class="topmenu">
    {{marktree|{{navtree}}|spec|{{spec}}}}
    {{foreach|{{navtree}}|a|
        {{if|{{a:isintrail}}|
            {{setnode|lev2node|{{a:handle}}}}|}}
        {{streq|{{a:type}}|I||
            {{if|{{a:isintrail}}|
                <b>{{a:name}}</b>
                |
                <a href="{{a:url}}">{{a:name}}</a>
            }}
        }}
    }}
    </div>
\end{MyVerbatim}
This is straightforward, except for the \usetag{setnode} lines: 
\begin{MyVerbatim}
        {{if|{{a:isintrail}}|
            {{setnode|lev2node|{{a:handle}}}}|}}
\end{MyVerbatim}
if the node is in the trail, store
its handle in \texttt{lev2node}. What this does is stores which node in the top level menu is
in our trail, because this is the menu we want to render for our second level. You might think
it would be a good idea to put this \texttt{setnode} inside the later in-trail test, rather
than doing two separate tests, but if you did that the system wouldn't work properly for hidden
pages, like the home page in our example: the set would never run for hidden nodes.

Our second level code follows on immediately afterwards:
\begin{MyVerbatim}
    <div class="2ndmenu">
    {{iftagexists|lev2node|
        {{iffieldexists|{{lev2node}}|child|
            {{foreach|{{lev2node|child}}|a|
                {{streq|{{a:type}}|I||
                    {{if|{{a:isintrail}}|
                        <b>{{a:name}}</b>|
                        <a href="{{a:url}}">{{a:name}}</a>
                    }}
                }}
            }}
        |}}
    |}}
    </div>
]]
\end{MyVerbatim}
This checks to see if the node we stashed away earlier was actually stored, and that it has
a child --- another menu. If so, we do exactly the same iteration to render the items in that
menu, using the syntax in section~\ref{fieldget} to get the value of the child field of the node,
which contains the collection handle of the menu we want.


\subsection{Pulldown menus}
\label{pulldownmenu}
This example is actually more about CSS and JavaScript than SCMS --- the SCMS code
is very similar to that which generates the simple menu with multiple levels, but the 
scripts make them look like a pulldown.
In a pulldown menu there are two levels: the root level, in which the items
are organised in a horizontal line across the page; and the second level, in
which items appear vertically below their parent. These items are normally
invisible until the parent is hovered over.

As I mentioned before, we can pretty much do this in CSS and JavaScript, given a hierarchy of
unordered list (UL) items. First we need to produce our menu system as a test
site in HTML, CSS and JavaScript with no clever SCMS stuff. I'm going to steal
some code from the Web
here\footnote{http://htmldog.com/articles/suckerfish/dropdowns/} and use it as
our basis. The HTML is a straightforward nested UL list, which appears in the
sample as this:
\begin{MyVerbatim2}
<ul id="nav">
	<li><a href="#">Percoidei</a>
		<ul>
			<li><a href="#">Remoras</a></li>
			<li><a href="#">Tilefishes</a></li>
			<li><a href="#">Bluefishes</a></li>
			<li><a href="#">Tigerfishes</a></li>
		</ul>
	</li>

	<li><a href="#">Anabantoidei</a>
		<ul>
			<li><a href="#">Climbing perches</a></li>
			<li><a href="#">Labyrinthfishes</a></li>
			<li><a href="#">Kissing gouramis</a></li>
			<li><a href="#">Pike-heads</a></li>
			<li><a href="#">Giant gouramis</a></li>
		</ul>
	</li>
</ul>
\end{MyVerbatim2}
I'm using a tiny font now because there's a lot of code here. Now we add some
clever CSS:
\begin{MyVerbatim2}
#nav, #nav ul { /* all lists */
	padding: 0;
	margin: 0;
	list-style: none;
	line-height: 1;
}

#nav a {
	display: block;
	width: 10em;
}

#nav li { /* all list items */
	float: left;
	width: 10em; /* width needed or else Opera goes nuts */
}

#nav li ul { /* second-level lists */
	position: absolute;
	background: orange;
	width: 10em;
	/* using left instead of display to hide menus because
	display: none isn't read by screen readers */
	left: -999em; 
}

#nav li:hover ul, #nav li.sfhover ul { /* lists nested under hovered list items */
	left: auto;
}
\end{MyVerbatim2}
And now we load this CSS from inside \texttt{main.html} in the template, along
with a little snippet of javascript for dumb browsers which don't support the
\texttt{hover} attribute. We use the \usetag{templateroot} tag to get the template
directory:
\begin{MyVerbatim2}
<link rel="stylesheet" href="{{templateroot}}site.css"/>
<script type="text/javascript">
<!--
sfHover = function() {
	var sfEls = document.getElementById("nav").getElementsByTagName("LI");
	for (var i=0; i<sfEls.length; i++) {
		sfEls[i].onmouseover=function() {
			this.className+=" sfhover";
		}
		sfEls[i].onmouseout=function() {
			this.className=this.className.replace(new RegExp(" sfhover\\b"), "");
		}
	}
}
if (window.attachEvent) window.attachEvent("onload", sfHover);
-->
</script>
\end{MyVerbatim2}
Finally, all we need is a very simple tag defined in the template's 
\texttt{menus.tags} or similar:
\begin{MyVerbatim2}
navmenupulldown=[[
{{treeprefix|default|<ul>}}
{{treesuffix|default|</ul>}}
{{treeprefix|0|<ul id="nav">}}

{{treeunselnode|default|<li>
    {{streq|{{a:type}}|H|
        {{a:name}}|
        <a href="{{a:url}}">{{a:name}}</a>
    }}|</li>
}}

{{treeselnode|default|<li><b>{{a:name}}</b>|</li>}}

{{marktree|{{navtree}}|spec|{{spec}}}}
{{rendertree|{{navtree}}|a}}
]]
\end{MyVerbatim2}
This is the same as our previous system, but we've embedded things a bit more
-- nesting some if the conditionals -- and we've put the \texttt{nav} CSS ID
on the outermost \texttt{UL}. It's very easy to use this as a basis for
prettier menus, for instance using different CSS classes for different levels
of the menu.

\section{Accessibility}
The navigation of a site's structure can difficult for users with some
disabilities. Most of these problems can be mitigated with 
a good HTML template design, and there are many resources on the Internet
to help with this. A good place to start is the W3C WCAG guidelines page
at \texttt{http://www.w3.org/TR/WCAG/}.

Two of the facilities which help accessibility are \emph{document relationship links}
and \emph{access keys}. SCMS provides methods for producing these.

\subsection{Document relationship links (LINK REL tags)}
\label{linkreltags}
\indtag{navlinks}
These are links which appear in the HEAD block of an HTML file, and have
the following form:
\begin{MyVerbatim}
<link rel="start" title="Home Page" href="http://foo.org/index.php/home"/>
<link rel="prev" title="Page 1" href="http://foo.org/index.php/page1"/>
<link rel="next" title="Page 3" href="http://foo.org/index.php/page3"/>
\end{MyVerbatim}
If you put the tag
\begin{MyVerbatim}
{{navlinks}}
\end{MyVerbatim}
in the HEAD block of your template, SCMS will insert a set of LINK REL
tags at that point which correctly match the structure of the navigation
menu and the current page's position within it. There's a lot of customisation which
can and should be done, which we'll learn about later.

\subsection{Access keys}
\indtagsec{navigation tree node fields!key}
Some sites rely on access keys to help users get to common pages. You can support this in your
menu by checking the \texttt{key} field in each node as you render it. This value is set from
the \texttt{navigation} file, by putting
\begin{MyVerbatim}[commandchars=+\[\]]
key=+emph[keyvalue]
\end{MyVerbatim}
after the navigation item, separated by a space. For example, if you wanted access key 1 on some
page\footnote{The
actual key combination pressed will be different for different browsers.
In Firefox, access key 1 would actually be ALT+SHIFT+1. In Internet Explorer, ALT+1 would
focus on the link, but then you'd have to press ENTER.} within your structure
you would have a line like:
\begin{MyVerbatim}
---page1 key=1
\end{MyVerbatim}
\indtagsec{navigation tree node fields!accesskeyattr}
If the \texttt{key} field is set, a special field called \texttt{accesskeyattr} will also be set inside
the node. Normally set to the empty string, it will now contain a string like
\begin{MyVerbatim}
accesskey="h"
\end{MyVerbatim}
You can use this field in your menu's tree rendering templates, as an attribute
inside the \texttt{A} tag. For example,
\begin{MyVerbatim}
{{treeunselnode|default|<li>
  <a href="{{a:url}}" {{a:accesskeyattr}}>{{a:name}}</a>
  |</li>}}
\end{MyVerbatim}

\subsubsection{Access keys for document relationship links (start, next, previous)}
In section~\ref{linkreltags}, we defined document relationship links. We can specify
access keys for these using \texttt{navlinks} tag, by passing in extra arguments --- one for
each key: start, next and previous. We also need to pass in a template
string, which is used to render the title text for the link, along with a prefix for the 
tags defined -- the link will have all the standard node fields available to it, from the node
that will be the target of the link.
For example, if we wanted to use the 1 key for the home
page, the N key for the next page, and the P key for the previous page, we could use
\begin{MyVerbatim}
{{navlinks|a|{{a:name}}, shortcut key {{a:lkey}}|1|N|P}}
\end{MyVerbatim}
Going through the arguments in order, we have:
\begin{itemize}
\item The prefix \texttt{a}, used to prefix the target node's fields when they're set up as tags,
in the same manner as a \texttt{foreach} or \texttt{withnode}.
\indtagsec{navigation tree node fields!lkey}
\item The actual template used to render the title of the link, given that the target node (i.e. the 
start, next or previous node) has had tags set up with the given prefix. Note that the \texttt{lkey}
field is now usable, giving the access key for the link (it stands for \emph{link key}.)
\item The access key for the first node.
\item The access key for the next node, if any.
\item The access key for the previous node, if any.
\end{itemize}
How this works is rather complex. Firstly, it causes the \texttt{LINK REL} tags output by \texttt{navlinks}
to have title strings rendered from the template, so we can do effective localisation --- particularly
of the ``shortcut key=xxx'' part (the default title is the name of the link.)
Secondly, it adds extra \texttt{accesskey} attributes to the 
\texttt{accesskeyattr} fields of the appropriate nodes, so that these nodes will have access keys
added to their links. These new keys will work alongside any keys you may have already defined with
the \texttt{key} option in the navigation file.

\subsubsection{Unused access keys}
If any access key is left empty, it won't be added. For example,
\begin{MyVerbatim}
{{navlinks|a|{{a:name}}, shortcut key {{a:lkey}}||N|P}}
\end{MyVerbatim}
does not have a start key assigned, so no start key will be set up.

\subsubsection{An example}
Here's the UK government's mandated scheme for access keys, applied to a navigation file:
\begin{MyVerbatim}
home key=1
-whatsnew key=2
-search key=4
-help key=6
--sitemap key=3
--faq key=5
--complaints key=7
--termsandconds key=8
--feedback key=9
--accesskeydata key=0
\end{MyVerbatim}
\clearpage
And here's a simple menu renderer, using the pulldown menu system of
section~\ref{pulldownmenu}:
\begin{MyVerbatim}
navmenupulldown=[[
{{treeprefix|default|<ul>}}
{{treesuffix|default|</ul>}}
{{treeprefix|0|<ul id="nav">}}

{{treeunselnode|default|<li>
    {{streq|{{a:type}}|H|
        {{a:name}}|
        <a href="{{a:url}}" {{a:accesskeyattr}}>{{a:name}}</a>
    }}|</li>
}}

{{treeselnode|default|<li><b>{{a:name}}</b>|</li>}}

{{marktree|{{navtree}}|spec|{{spec}}}}
{{rendertree|{{navtree}}|a}}
]]
\end{MyVerbatim}
And here's what we'll put into the main template. Note that we're not using a start page access key,
because that's already dealt with by the navigation file's explicit key assignment in the first line:
\begin{MyVerbatim}
{{navlinks|a|{{a:name}}, shortcut key {{a:lkey}}||N|P}}
\end{MyVerbatim}

\section{Creating links within the site}
\indtag{url}
To specify links to other pages which aren't part of menus but in the
body of the text, use the \texttt{url} tag to generate a URL given a 
page specification:
\begin{MyVerbatim}
<a href="{{url|contact}}">Contact page</a>
<a href="{{url|{{defaultpage}}}}">Home page</a>
\end{MyVerbatim}
Note the use of \usetag{defaultpage} to get the specification of the default
page (i.e. the home page.)

An optional second argument is the \emph{style name}, if you're using
styles (see chapter~\ref{styles}.)

If you're using HTTP GET parameters, these will be added on to any
URLs created with the \texttt{url} tag. If you don't want this,
use the \usetag{urlnoget} tag instead, which discards all of these.
You will then have to hand on any tags you want to keep explicitly.

\indtag{link}
The \texttt{link} tag is also useful. It takes the page specifier,
a link text and an optional style; and generates a full \texttt{A HREF}
link tag.
