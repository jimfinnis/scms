%%%%%%%%%%%%%%%%%%%%%%%%%%%%%%%%%%%%%%%%%%%%%%%%%%%%%%%%%%%%%%%%%%%%%%%%%%%%%
%
%                     Copyright 2009 Jim Finnis.
%                           All Rights Reserved
%
%
%  System        : 
%  Module        : 
%  Object Name   : $RCSfile$
%  Revision      : $Revision$
%  Date          : $Date$
%  Author        : $Author$
%  Created By    : Jim Finnis
%  Created       : Tue Dec 29 17:04:53 2009
%  Last Modified : <110102.2015>
%
%  Description 
%
%  Notes
%
%  History
% 
%%%%%%%%%%%%%%%%%%%%%%%%%%%%%%%%%%%%%%%%%%%%%%%%%%%%%%%%%%%%%%%%%%%%%%%%%%%%%
%
% Copyright (c) 2009 Jim Finnis.
% 
% All Rights Reserved.
% 
% This  document  may  not, in  whole  or in  part, be  copied,  photocopied,
% reproduced,  translated,  or  reduced to any  electronic  medium or machine
% readable form without prior written consent from Jim Finnis.
%
%%%%%%%%%%%%%%%%%%%%%%%%%%%%%%%%%%%%%%%%%%%%%%%%%%%%%%%%%%%%%%%%%%%%%%%%%%%%%

\chapter{Reference}
\section{Tag Definition File Syntax}
This is a description of the syntax of tag files in BNF, using the following conventions:
\begin{itemize}
\item Optional items are enclosed in \texttt{[square brackets]}. 
\item Repeating items (i.e. 1..$n$ copies) are enclosed in \texttt{\{curly brackets\}}.
\item An optional repeating item (i.e. 0..$n$ copies) is enclosed in both types of bracket.
\item Terminals (i.e. literal strings) are enclosed in `single quotes'
\item \texttt{<string>} is a valid ASCII string containing tags (see Section~\ref{reftagsyn}, Tag
Syntax)
\item \texttt{<whitespace>} is a whitespace character, i.e. a space or tab
\end{itemize}
\begin{MyVerbatim}
<file> ::= {[<definition>]}

<definition> ::= <multilinedef>|<singlelinedef>

<singlelinedef> ::= <tagname> {[<whitespace>]} `='
                    {[<whitespace>]} <string> `\n'

<multilinedef> ::= <tagname> {[<whitespace>]} `='
                    {[<whitespace>]} `[[' {[<whitespace>}] `\n'
                    {<string>} `]]' {[<whitespace>]} `\n'

\end{MyVerbatim}


\section{Tag Syntax}
\label{reftagsyn}
Tags in strings have the following syntax (using the same BNF conventions as the previous section):
\begin{MyVerbatim}
<tag> ::= `{' <tagname> [{ `|' <argument> }] `}'
<tagname> ::= {<alphanumeric character>}
<argument> ::= <string>
\end{MyVerbatim}

\section{Tags}
In the list below, any function tag arguments which are \emph{not} processed
through the templating system and so do \emph{not} have any embedded
tags expanded are marked with \verb+*+ .

Optional arguments are in brackets.
\input{tmpref}
\scriptsize
\clearpage
\printglossary
\clearpage
\printindex{tags}{An index of tags and fields}
