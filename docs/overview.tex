%%%%%%%%%%%%%%%%%%%%%%%%%%%%%%%%%%%%%%%%%%%%%%%%%%%%%%%%%%%%%%%%%%%%%%%%%%%%%
%
%                     Copyright 2010 Jim Finnis.
%                           All Rights Reserved
%
%
%  System        : 
%  Module        : 
%  Object Name   : $RCSfile$
%  Revision      : $Revision$
%  Date          : $Date$
%  Author        : $Author$
%  Created By    : Jim Finnis
%  Created       : Sun Jan 3 15:35:20 2010
%  Last Modified : <110116.1259>
%
%  Description 
%
%  Notes
%
%  History
% 
%%%%%%%%%%%%%%%%%%%%%%%%%%%%%%%%%%%%%%%%%%%%%%%%%%%%%%%%%%%%%%%%%%%%%%%%%%%%%
%
% Copyright (c) 2010 Jim Finnis.
% 
% All Rights Reserved.
% 
% This  document  may  not, in  whole  or in  part, be  copied,  photocopied,
% reproduced,  translated,  or  reduced to any  electronic  medium or machine
% readable form without prior written consent from Jim Finnis.
%
%%%%%%%%%%%%%%%%%%%%%%%%%%%%%%%%%%%%%%%%%%%%%%%%%%%%%%%%%%%%%%%%%%%%%%%%%%%%%

\chapter{Overview}
This is the \emph{real} documentation, going into a lot of detail.
Please read the previous chapter first.
\section{Installing SCMS}
\label{installation}
You will need the following:
\begin{itemize}
\item Some degree of knowledge of PHP, but not too much.
 \item A server running PHP5 (I'm using 5.2 to test with).
\item Shell access on that server.
\end{itemize}

And here are the steps to follow:
\begin{itemize}
\item First, check SCMS out of the repository, or unpack the SCMS archive,
into a directory called \texttt{scms} on your server, but not your public HTML
directory.
\item Link the \texttt{core} directory into your public HTML directory, where
you want your new site to be. If you can't link, it's OK to copy it, but make
sure you copy the whole core directory structure (\texttt{cp -R} is your
friend.)
\useglosentry{coredir}

\item Link the \texttt{core/idx.php} file to \texttt{index.php} in your public HTML directory. Again, it's OK to copy if you have an operating system that doesn't do easy linking.
\item While in the destination directory, run the \texttt{createdirs} script to create the temporary directories and set the correct permissions. You may find the permissions somewhat over-permissive;
\useglosentry{createdirs}
feel free to modify them --- SCMS must be able to create and write files in the \texttt{tmp} and \texttt{tmp/cache} directories.
\item Copy (not link) the \texttt{default\_config.php} directory into the
new directory \texttt{site} inside your destination directory, and call it \texttt{config.php}.
\useglosentry{sitedir}
\item Link or copy the files for the languages you require from \texttt{NLSFiles\footnote{It stands for National/Native Language System}} into the new \texttt{site/languages} directory.
For example, if you need English and Welsh, use \texttt{en\_US.nls.php} and \texttt{cy\_GB.nls.php}. If your language isn't included, it's easy to create a new NLS file --- see the files that are
there for examples.
Note that if you ever add a new language, you'll have to clear the cache
by running the \texttt{clearcache} script in the SCMS directory, from your
root directory.
\useglosentry{nls}
\item Copy or link at least the \texttt{default} template directory from the sample templates into the new \texttt{site/templates} directory. The default template is minimal, you'll be either editing it
or replacing it with another template.
\item Create a default home page inside \texttt{site/pages}. You can copy the \texttt{testhome} file as a starting point.
\item Create a \texttt{site/navigation} file indexing this page. This can just consist of the word \texttt{home} on a line by itself, indicating that the site has one page, called `home.'
\item Modify the \texttt{site/config.php} file you copied earlier. See below for notes on how to modify it.
\item You may wish to create a 404 template and page, and add modules, but we'll deal with those later. At this point your site should work, showing the default page successfully.
\item Now you can build the rest of your site by modifying or creating new templates and pages.
\end{itemize}
\useglosentry{configphp}

\subsection{Modifying the config.php file}
\label{modiconfig}
This is a PHP file containing several values you may need to change, though
many sites won't require it to start with. Later on, you will need to change
the caching data:
\begin{itemize}
 \item \textbf{ROOTPATH} gives the absolute path of the root directory of the site in the server's filesystem --- i.e. where your index.php, site and core directories are. It's
 generated automatically from server variables (i.e. those in \verb,$_SERVER[],) but you can set it by hand if this
 fails for some reason.
 \item \textbf{ROOTURL} gives the root URL of the site, \textbf{without} the index.php. It's generated automatically
 from server variables, but again you can set it by hand.
\item \textbf{DEFAULTLANG} is the code for the default language. You must have the corresponding NLS file in your \texttt{site/languages} directory.
For example, if you set DEFAULTLANG to \texttt{en\_GB}, you must have \texttt{en\_GB.nls.php} in there. By default, it's set to \texttt{en\_US}: US English.
\item \textbf{DEFAULTPAGE} is the name of the default page to use when no pagename is specified. By default, it's `home.'
\item \textbf{CACHING} is used to enable page caching. Set it to 0 when testing and 1 when your site is ready.
\item \textbf{BROWSERCACHING}, when turned on, makes sure most pages are cached on the browser. See chapter~\ref{cachechap} for details. Set to 0 when testing and 1 when your site is ready.
\item \textbf{\$cache\_exceptions} is an array of strings, giving the names of pages which should not be cached, either on the server or in the browser.
It's empty by default, but you should add to it the names of pages which provide interactive content.
\end{itemize}
Certain modules may add variables to the configuration file --- see each module you use for documentation.

\section{The SCMS directory}
Here's the top level of a typical SCMS directory. It's this directory which will contain the \texttt{index.php} file, through which all SCMS pages are accessed. This
is usually a link to \texttt{core/idx.php}.
Those entries in \emph{\texttt{italics}} are directories, the rest are files.
\begin{center}
% use packages: array
\begin{tabular}{lll}
\texttt{clearcache} & \emph{\texttt{core}} & \texttt{index.php}  \\ 
\emph{\texttt{modules}} & \emph{\texttt{site}}
\end{tabular}
\end{center}

\begin{itemize}
 \item \textbf{clearcache} is a short script to clear the page cache and temporary files.
You should run this after modifying the navigation or language data (see below).
\item \textbf{core} is a directory containing the core files of SCMS. Usually, it's actually a link to a shared directory
for all sites managed on this server.
\item \textbf{index.php} is the main SCMS PHP source file. You shouldn't need to alter it after
installation. In fact, it's usually a link to a file in the \emph{core} directory.
\item \textbf{modules} is a directory containing all the modules available with SCMS by default.
In general, you link to modules you require in this directory (see below). Again, it's often just a link
into the standard SCMS distribution (but see section~\ref{alternatemod}.)
\item \textbf{site} is a directory which contains all the information specific to your site.
This is the only directory in which you should need to change anything.
\end{itemize}

\clearpage
\section{The \emph{site} directory}
\useglosentry{sitedir}
This directory, automatically created along with its subdirectories with the
\texttt{createdirs} script, is where you create your site files. It's the only
directory whose contents you modify after the initial installation. It looks
like this:
\begin{center}
% use packages: array
\begin{tabular}{lll}
\texttt{config.php} & \emph{\texttt{global}} & \emph{\texttt{languages}}  \\ 
\emph{\texttt{modules}} & \texttt{navigation} & \emph{\texttt{pages}} \\
\emph{\texttt{templates}}
\end{tabular}
\end{center}
This breaks down thus:
\begin{itemize}
 \item \texttt{config.php} is the site configuration file, which you must modify.
\item \texttt{global} contains global tag definition files\footnote{files which define tags which are expanded by the templating system --- see below.} which are loaded for all pages, and which define tags which can be used in all templates and pages.
\item \texttt{languages} contains the NLS files for each language the site can use.
\item \texttt{modules} contains a directory for each module which can be used in the site. These are typically copied or linked from the installation \texttt{modules} directory
 (but see section~\ref{alternatemod}.)
\item \texttt{navigation} is a file describing the structure of the site as a whole: how the pages are related to each other. It's used to generate the navigation menus.
\item \texttt{pages} contains a single tag definition file for each page in the site. Each file has the same name as the page name as it appears in the URL.
\item \texttt{templates} contains a directory for each template. Each page either specifies the template it uses, or the \texttt{default} template is used.
\end{itemize}

\subsection{Alternate module structure}
\label{alternatemod}
If a site only uses standard modules, there may be no \texttt{modules} directory at the top level. Instead, \texttt{site/modules} may link directly
to the SCMS distribution's \texttt{modules} directory, which contains all the standard modules.



\section{Anatomy of an SCMS URL}

The URL of page on an SCMS site is divided into two parts: the \textbf{root URL}
and the \textbf{page specifier}. The root URL is simply the URL of the site
as a whole, up to `index.php.' for example
\begin{MyVerbatim}
http://foo.com/index.php
\end{MyVerbatim}
or
\begin{MyVerbatim}
http://foo.com/blog/index.php
\end{MyVerbatim}
Note that \texttt{index.php} cannot be omitted unless you're accessing the home page (the first page on the top
level of the navigation file.)\footnote{Note, however, that the \texttt{ROOTURL} value defined or generated in \texttt{config.php}
is the URL \emph{without} the \texttt{index.php}}

\useglosentry{pagespec}
The \textbf{page specifier} is the rest of the URL after \texttt{index.php}. It consists of the page name
and any GET query details: \texttt{home} or \texttt{prods/chair?q=hello\&f=bar}.
You can have slashes in the page specifier, and this will imply a directory structure in your
pages directory (see below.) If there is no page specifier, the default page (set in \texttt{config.php}) is assumed.

\subsection{The canonical URL}
\useglosentry{canonurl}
Most links created on other sites will be in the form given above, with or without the
\texttt{index.php} in the case of the home page.
However, an internal link within the site will be in the \emph{canonical form}, which incorporates
the language code. This is the same as the normal URL, but with the language code between the root
URL and the page specifier. For example \texttt{http://foo.com/index.php/en\_GB/home}. 
\subsection{Example}
Here's an example of a SCMS URL, and a canonical form of the same URL, given that the default language is US
English:
\begin{MyVerbatim}
	http://foo.com/index.php/rooms/room1
	http://foo.com/index.php/en_US/rooms/room1
\end{MyVerbatim}
This means SCMS should serve the page whose specifier is \texttt{rooms/room1}.

