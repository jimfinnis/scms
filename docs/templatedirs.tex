%%%%%%%%%%%%%%%%%%%%%%%%%%%%%%%%%%%%%%%%%%%%%%%%%%%%%%%%%%%%%%%%%%%%%%%%%%%%%
%
%                     Copyright 2010 Jim Finnis.
%                           All Rights Reserved
%
%
%  System        : 
%  Module        : 
%  Object Name   : $RCSfile$
%  Revision      : $Revision$
%  Date          : $Date$
%  Author        : $Author$
%  Created By    : Jim Finnis
%  Created       : Tue Jan 19 17:06:14 2010
%  Last Modified : <100824.1641>
%
%  Description 
%
%  Notes
%
%  History
% 
%%%%%%%%%%%%%%%%%%%%%%%%%%%%%%%%%%%%%%%%%%%%%%%%%%%%%%%%%%%%%%%%%%%%%%%%%%%%%
%
% Copyright (c) 2010 Jim Finnis.
% 
% All Rights Reserved.
% 
% This  document  may  not, in  whole  or in  part, be  copied,  photocopied,
% reproduced,  translated,  or  reduced to any  electronic  medium or machine
% readable form without prior written consent from Jim Finnis.
%
%%%%%%%%%%%%%%%%%%%%%%%%%%%%%%%%%%%%%%%%%%%%%%%%%%%%%%%%%%%%%%%%%%%%%%%%%%%%%

\chapter{Template Directories}
\label{templatedirs}
\useglosentry{tempdir}
When you create your site, the first thing you will need to do is create a
template. SCMS does provide a default template, which will be used for any
page which doesn't explicitly specify one, but it's extremely minimal -- you
should replace it with your own as soon as possible.

Each template is contained its own directory in the \texttt{site/templates}
directory, the name of which is the name of the template. It contains at least
the file \texttt{main.html}, and probably a few \texttt{.tags} files. It
should also contain any other files used by that template --- CSS and
ECMAScript files, images and so forth. These can be in subdirectories if you
like.

\section{How templates work}
As you'll see later, each page is actually a file in tag definition language, each tag defined being
automatically prefixed by `\texttt{page:}' as we mentioned earlier. The page says which template it
should be rendered with, and the template defines what the overall `shape' of the page is. This is
how it works:

After the page file is read in,
the \texttt{page:template} tag is checked. If it doesn't exist, we'll use the \texttt{default} template,
otherwise we'll use the template named.

We then read any \texttt{.tags} files in the template's directory, parsing them to define their tags. Such
tags will be prefixed with `\texttt{template:}'

Finally we read in the \texttt{main.html} template file, and put it through the
template processor. This will use the tags defined in both the template and the page to generate HTML
output. The output is then fed to the client browser\footnote{Naturally I'm
missing out a lot here: for example how \texttt{globals} is processed, and how modules and caching work.}.

\section{The main template file}
\useglosentry{mainfile}
This file, called \texttt{main.html}, is the starting point for any page
output. It's a plain HTML file containing template tags which are replaced to
generate the page. The output produced is then fed to the client browser.

\subsection{An example \texttt{main.html}}
\label{samplemain}
Here's an example of such a file:
\begin{MyVerbatim}
<html>
 {{scmstag}}
 <head>
  <title>{{page:name}}</title>
  <link rel="stylesheet" href="{{templateroot}}site.css"/>
  {{navlinks}}
 </head>
 <body>
   <h2>{{page:name}}</h2>
   {{template:langmenu}}
   <h3>Navigation</h3>
   {{template:navmenu}}
   <p>Time created: {{ftime|%c}}</p>
   {{@page:content}}
 </body>
</html>
\end{MyVerbatim}
The following tags are used in this template:
\begin{itemize}
\indtag{scmstag}
\item \texttt{scmstag} outputs an HTML comment which says that this HTML 
was generated by SCMS, and which version. It typically looks like this:
\begin{MyVerbatim}
<!-- Generated by SCMS, $Rev: 305 $ $Date: 2010-08-24
    15:30:17 +0100 (Tue, 24 Aug 2010) $ (of functpls) -->
\end{MyVerbatim}
\item \texttt{page:name} is the actual name of the page, defined in page's tag definition file.
\indtag{templateroot}
\item \texttt{templateroot} produces the URL for the current template directory (followed by a slash),
which is where we would expect to find data for the template, like images and CSS files.
\indtagsec{navlinks}
\item \texttt{navlinks} is a convenient tag defined by the system which produces accessibility
\texttt{<link rel>} HTML tags defining the page's relationship to others in the site --- see 
Chapter \ref{navig}: Navigation, for more details.
\item \texttt{template:langmenu} is a tag defined in one of the template's own \texttt{.tags} files,
describing how to produce the languages menu.
\item \texttt{template:navmenu} is a tag defined in one of the template's \texttt{.tags} files, describing
how to produce the navigation menu.
\indtag{ftime}
\item \texttt{ftime} is a system tag to output the current time (i.e. the time at which the page is
generated) using the C library's \texttt{strftime()} function. The argument is the format to
use - \texttt{\%c} means the form `Tue Jan 5 18:39:44 2010.' You can use this to determine how old a page
is in the cache, if caching is enabled.
\item \texttt{page:content} is a tag defined by the page, which is the actual content of the page. The
value of this tag is likely to be a large, multiline string which may contain other tags to be expanded.
The `@' means that if the tag doesn't exist, an empty string should be used instead of giving an error.
\end{itemize}

\section{The .tags files}
\useglosentry{tagsfile}
\useglosentry{templatetag}
As noted above, SCMS scans the template directory for any tag definition files
ending in \texttt{.tags} and reads them in, defining the tags and prepending
their names with the \texttt{template:} string. Tags thus defined are called
\emph{template tags}.

These tags are mainly used in the \texttt{main.html} file, defining such things as how
the page renders its menus. Here's an example which fits with the example file
above. Don't worry too much about it for now, it'll be
explained by Chapter \ref{trees}: Trees and Menus.

\indtagsec{treeprefix}
\indtagsec{treesuffix}
\indtagsec{treeselnode}
\indtagsec{treeunselnode}
\indtagsec{marktree}
\indtagsec{rendertree}
\begin{MyVerbatim}
navmenu=[[
{{treeprefix|default|<ul>}}
{{treesuffix|default|</ul>}}
{{treeselnode|default|
    <li class="currentpage">
    <h3>{{item:name}}</h3>|</li>}}
{{treeunselnode|default|
    <li><a href="{{item:url}}" title="{{item:name}}">
        {{item:name}}</a>|</li>}}
{{marktree|{{navtree}}|spec|{{spec}}}}
{{rendertree|{{navtree}}|item}}
]]

langmenu=[[
{{treeprefix|default|<small>}}
{{treesuffix|default|</small>}}
{{treeselnode|default|{{a:endonym}}|}}
{{treeunselnode|default|<a href="{{a:url}}">{{a:endonym}} </a>|}}
{{marktree|{{langtree}}|code|{{langcode}}}}
{{rendertree|{{langtree}}|a}}
]]
\end{MyVerbatim}

This file defines two tags: \texttt{template:navmenu} and \texttt{template:langmenu}, which render
the navigation and language menus respectively. To understand what the various tags do, you'll need to
look at Chapter, \ref{trees}: Trees and Menus. One thing that's worth noting is that a
lot of the tags --- \texttt{treeselnode} and so forth --- don't actually produce
any text. Instead, they just set strings inside SCMS' menu system that are used when the \texttt{rendertree}
tag runs. There are quite a few tags which set internal states rather than producing text.	

