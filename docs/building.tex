%%%%%%%%%%%%%%%%%%%%%%%%%%%%%%%%%%%%%%%%%%%%%%%%%%%%%%%%%%%%%%%%%%%%%%%%%%%%%
%
%                     Copyright 2010 Jim Finnis.
%                           All Rights Reserved
%
%
%  System        : 
%  Module        : 
%  Object Name   : $RCSfile$
%  Revision      : $Revision$
%  Date          : $Date$
%  Author        : $Author$
%  Created By    : Jim Finnis
%  Created       : Mon Jan 11 16:05:51 2010
%  Last Modified : <110116.1259>
%
%  Description 
%
%  Notes
%
%  History
% 
%%%%%%%%%%%%%%%%%%%%%%%%%%%%%%%%%%%%%%%%%%%%%%%%%%%%%%%%%%%%%%%%%%%%%%%%%%%%%
%
% Copyright (c) 2010 Jim Finnis.
% 
% All Rights Reserved.
% 
% This  document  may  not, in  whole  or in  part, be  copied,  photocopied,
% reproduced,  translated,  or  reduced to any  electronic  medium or machine
% readable form without prior written consent from Jim Finnis.
%
%%%%%%%%%%%%%%%%%%%%%%%%%%%%%%%%%%%%%%%%%%%%%%%%%%%%%%%%%%%%%%%%%%%%%%%%%%%%%

\chapter{Building a Site}
This chapter outlines the process of building a site from scratch,
and gives references to other chapters to help you.

\section{Building the starting site}
Your artist should have produced a static HTML site for you, consisting
of one HTML page with some placeholder content. Our first step is to
turn that into an SCMS site, using the static HTML as a template.
\subsection{Install SCMS}
Naturally, if you haven't installed SCMS you'll need to do that --- typically
into a directory \emph{outside} the public HTML directory,
which you then create links to. This allows several sites to share the same
SCMS installation. This is described on page~\pageref{installation}.

\subsection{Create the directories for your site}
You'll also need to create the public HTML directories for each site you're
going to set up. This is described more fully in section~\ref{installation},
but it basically consists of:
\begin{itemize}
\item creating the public HTML directory if it's not already there --- 
usually the hosting service sets one up --- from now on I'll refer
to this as the \emph{root directory};
\item creating a link to the SCMS \texttt{core} directory inside the
root directory;
\item creating a link to \texttt{core/idx.php} named \texttt{index.php} 
inside the root directory;
\item running \texttt{scms/createdirs} from inside the root
directory;
\item copying \texttt{scms/default\_config.php} into the \texttt{site}
directory and renaming it \texttt{config.php};
\item editing \texttt{config.php} as described on page~\pageref{modiconfig}
(chances are you won't need to do anything;)
\item copying the language files required into the the \texttt{site/languages}
directory. These can be 
found in \texttt{NLSfiles} in the SCMS distribution.
Note that if you ever add a new language, you'll have to clear the cache
by running the \texttt{clearcache} script in the SCMS directory, from your
root directory.
\end{itemize}
You'll now have the following things in your root directory:
\begin{itemize}
\item \texttt{site} is where your site's content lives;
\item \texttt{site/languages} contains the definition files for the languages
your site understands;
\item \texttt{site/pages} contains files for each individual page;
\item \texttt{site/templates} contains a directory for each template your site uses;
\item \texttt{site/global} contains global tag definitions (often this is empty;)
\item \texttt{tmp} is a temporary directory;
\item \texttt{tmp/cache} contains server-side caching data;
\item \texttt{config.php} is the configuration file.
\end{itemize}

\subsection{Creating the site template}
Your web designer should have created a static HTML and CSS page which
you can use as a template, consisting of the elements common to all
sites with some placeholder content. This should consist of a single
HTML file, one or more CSS files, some images, and maybe some Javascript files.

Copy these files into a \texttt{templates/default} directory (which you'll
need to create,) renaming the HTML file as \texttt{main.html}.

Edit \texttt{main.html} so that the names of all external files --- images,
CSS and Javascript --- are changed to
\begin{MyVerbatim}
{{templateroot}}myfilename.ext
\end{MyVerbatim}
The \texttt{{{templateroot}}} tag will be changed to the root directory
of the current template, i.e. \texttt{templates/default/}. Don't worry
about the placeholder content yet.

\subsection{Creating a dummy page and navigation file}
Create a file called \texttt{home} in the \texttt{site/pages} subdirectory,
with the following content:
\begin{MyVerbatim}
name=The Home Page
\end{MyVerbatim}

Now create a file called \texttt{navigation} in the \texttt{site} directory,
with a list of our pages. Naturally, it's just
\begin{MyVerbatim}
home
\end{MyVerbatim}
for now. The process of installing SCMS and setting up a site, along with the structure
of the site and how the URLs work, is described in more detail in the overview, section~\ref{installation}.

\subsection{Test it!}
You should now be available, and you should see the site your artist
designed.

\section{Templating}
The trick now is to add different content for each page using templating.
As an example, edit the \texttt{main.html} file changing all occurrences
of the page's title to the string
\begin{MyVerbatim}
{{page:name}}
\end{MyVerbatim}
Now when you load your page, you'll see that string is changed to
\begin{MyVerbatim}
The Home Page
\end{MyVerbatim}
which is the value of the \texttt{page:name} tag for that page, as set in
the page's file.

What you can do now is replace the placeholder content in \texttt{main.html}
with \verb+{{page:content}}+, and then create some content inside 
\texttt{site/pages/home} using a multi-line tag definition:
\begin{MyVerbatim}
name=The Home Page
content=[[
    <p>This is the home page. Please put something in here
    which makes more sense.</p>
]]
\end{MyVerbatim}

\subsection{Template dumping for debugging}
For fun, change the definition of your \texttt{home} page to this:
\begin{MyVerbatim}
name=The Home Page
content={{dump}}
\end{MyVerbatim}
If you reload the page, you'll get a full listing of all the tags currently defined\footnote{assuming
your template's CSS can handle wide tables well!}. Most of these
will be internal to SCMS, but two will be yours --- scan down to see them. Each entry in the list shows:
\begin{itemize}
\item the tag name,
\item the tag type (typically text or a function, with the name of the PHP function which performs the
tag if the latter),
\item an optional comment,
\item where the tag is defined: \texttt{(internal)} if SCMS defines it, or a tag definition file name if
you've defined it in your site.
\end{itemize}
\subsection{Function tags}
As noted above, some tags are functions. When using a function tag, the arguments are separated
from each other, and from the tag name, using the vertical bar character $\vert$. 
Try changing your \texttt{home} page file to this:
\begin{MyVerbatim}
name=The Home Page
content=[[
    Your wibble tag was : {{httpgetparam|wibble}}
]]
\end{MyVerbatim}
Now, if you append \texttt{?wibble=foo} to your browser URL when you load your site, you'll
get a page telling you what the value of the \texttt{wibble} HTTP GET value was. We can also check
that a value was set:
\begin{MyVerbatim}
name=The Home Page
content=[[
    Your wibble tag was : {{ifnotempty|{{httpgetparam|wibble}}|
                                {{httpgetparam|wibble}}|
                                <i>(not defined)</i>
                          }}
]]
\end{MyVerbatim}
Here, the \texttt{ifnotempty} tag takes three arguments --- the first is the string we're checking,
the second is the string to return if the first argument is longer than zero length, and the third
is the string to return otherwise. It should be obvious what the code does and how it works.
Note that the spacing and newlines aren't important, since they'll expand to HTML spaces which are
generally ignored.

There's a lot more about templating in chapter~\ref{templatesandtags} on page~\pageref{templatesandtags},
which describes how the Template Definition Language works in detail. Also, chapter~\ref{templatedirs}
on page~\pageref{templatedirs} will explain more about the structure of template directories,
and chapter~\ref{chappages} on page~\pageref{chappages} will explain how pages work.

We can now create several pages by adding entries for each page to the \texttt{navigation} file
and creating new page files inside \texttt{site/pages}, each of which has the same name as the entry
in the navigation file and defines the \texttt{page:} tags expected by the template in a different way.

\subsection{Page URLS}
The URLs for these new non-default pages are in the following form:
\begin{MyVerbatim}
http://mysite.blah.com/index.php/pagename
\end{MyVerbatim}
In other words, the page name is bolted onto the URL after \texttt{index.php}.
We could write \texttt{A HREF} tags by hand to create intra-site links, but
there is a better way.

\subsection{Links between pages}
To create links between pages, you can use the \texttt{url} tag to generate a full URL for a given
page. For example, a link to a page called \texttt{misc} can be created by:
\begin{MyVerbatim}
<a href="{{url|misc}}">Miscellaneous</a>
\end{MyVerbatim}

\section{Navigation}
This still isn't ideal -- we don't want to have to create separate navigation menus on every 
page. Instead, we use the templating system's complex collection handling to create and render
a navigation menu according to our requirements.

Replace your home page with this code:
\begin{MyVerbatim}
name=The Home Page

menu=[[
    {{treeprefix|default|<ul>}}
    {{treesuffix|default|</ul>}}
    {{treeunselnode|default|<li>
        <a href="{{a:url}}">{{a:name}}</a>|</li>}}
    {{treeselnode|default|<li>{{a:name}}|</li>}}
    {{marktree|{{navtree}}|spec|{{spec}}}}
    {{rendertree|{{navtree}}|a}}
]]

content=[[
{{page:menu}}
<p>My test content</p>
]]
\end{MyVerbatim}
This is the simplest possible navigation menu, which renders as a simple list of items. All pages
are rendered as links, apart from the current one. Don't worry about how it works internally ---
all it does is set up a bunch of templates which \texttt{rendertree} uses to draw a tree of page
navigation nodes.

What you should notice is that we've set up a \texttt{menu} tag inside the page, and invoked it
from the page's content using \texttt{page:menu}. All tags defined in page files are prefixed
automatically with \texttt{page:}.

\subsection{Putting the menu definition in the template}
We don't want to have to use this code for every page we create --- it should really be part of
the template. We can do this by adding a new \emph{tag file} to the \texttt{templates/default}
directory. Let's call it \texttt{menu.tags} and give it the following content:
\begin{MyVerbatim}
menu=[[
    {{treeprefix|default|<ul>}}
    {{treesuffix|default|</ul>}}
    {{treeunselnode|default|<li>
        <a href="{{a:url}}">{{a:name}}</a>|</li>}}
    {{treeselnode|default|<li>{{a:name}}|</li>}}
    {{marktree|{{navtree}}|spec|{{spec}}}}
    {{rendertree|{{navtree}}|a}}
]]
\end{MyVerbatim}
In other words, just the menu code. Any file in the current template's directory
which ends with \texttt{.tags} will automatically get loaded and its tags defined.

We can now replace our page file's content with:
\begin{MyVerbatim}
name=The Home Page

content=[[
{{template:menu}}
<p>My test content</p>
]]
\end{MyVerbatim}
Note that every tag defined in a template's \texttt{.tags} file is prefixed with \texttt{template:}.

\subsection{Putting the whole menu in the template}
We can do even better than this, of course. What we really should do is put \verb+{{template:menu}}+
somewhere inside our \texttt{main.html} and remove all references to the menu from our page files.
That way, we define the menu once and never have to worry about it again. 

Your web designer will probably have put some kind of navigation menu into his static HTML file --- you'll
need to work out how to convert this into a menu definition like the one above, which can be quite 
complex. Chapter~\ref{navig} will give you some help with this.

\section{More!}
There's a lot more in SCMS, including:
\begin{itemize}
\item hierarchical navigation trees (see chapter~\ref{navig},)
\item multiple templates for different pages (just add a \texttt{template=} line to your page file, see
chapter~\ref{chappages},)
\item multilingual support (see chapter~\ref{multil},)
\item complex collection handling (see chapters \ref{collchap} and \ref{trees}) --- you can use
this in modules to create lists or hierarchies of nodes for your page to render the same way as navigations
are rendered;
\item a standard system for adding new tags and GET and POST handlers (see chapter~\ref{moduleschap},)
\item a 404 handling system (see chapter~\ref{404chap},)
\item styles for different devices (see chapter~\ref{styles},)
\item site caching, so the system doesn't regenerate a page's content unless it's been changed (see
chapter~\ref{cachechap}.)
\end{itemize}

