%%%%%%%%%%%%%%%%%%%%%%%%%%%%%%%%%%%%%%%%%%%%%%%%%%%%%%%%%%%%%%%%%%%%%%%%%%%%%
%
%                     Copyright 2010 Jim Finnis.
%                           All Rights Reserved
%
%
%  System        : 
%  Module        : 
%  Object Name   : $RCSfile$
%  Revision      : $Revision$
%  Date          : $Date$
%  Author        : $Author$
%  Created By    : Jim Finnis
%  Created       : Tue Jan 19 17:06:02 2010
%  Last Modified : <110102.1929>
%
%  Description 
%
%  Notes
%
%  History
% 
%%%%%%%%%%%%%%%%%%%%%%%%%%%%%%%%%%%%%%%%%%%%%%%%%%%%%%%%%%%%%%%%%%%%%%%%%%%%%
%
% Copyright (c) 2010 Jim Finnis.
% 
% All Rights Reserved.
% 
% This  document  may  not, in  whole  or in  part, be  copied,  photocopied,
% reproduced,  translated,  or  reduced to any  electronic  medium or machine
% readable form without prior written consent from Jim Finnis.
%
%%%%%%%%%%%%%%%%%%%%%%%%%%%%%%%%%%%%%%%%%%%%%%%%%%%%%%%%%%%%%%%%%%%%%%%%%%%%%

\chapter{Collections}
\label{collchap}
\useglosentry{collec}
A collection is an array of data structures, with a fixed set of fields.
For example, a collection of news items could have the fields `date' and `text.'

\section{Simple collections, and iterating with foreach}
\useglosentry{item}
Simple collections only have one field, whose name is `value'.
To create such a collection in Tag Definition Language, define a tag using `+=' instead
of just an equals sign. If a collection of that name does not exist, it is created.
A new item is then created with the string specified as its `value' field, and added to the
collection. Naturally, if a tag already exists with that
name which is not a collection, an error occurs. Here's an example,
using the \texttt{foreach} tag to iterate over the collection:
\indtagsec{foreach}

\begin{MyVerbatim}
 mycollection += first item
 mycollection += second item
 mycollection += [[And this is the third item, which is 
   a multiline item]]

 ## here we use foreach, so that {{page:listofitems}} expands to
 ## an unordered HTML list of the items

 listofitems = [[
     <ul>
     {{foreach|{{page:mycollection}}|a|
         <li>{{a:value}}</li>
     }}
     </ul>
]]
\end{MyVerbatim}
\indtag{collection item fields!value}
In this example, the \texttt{foreach} tag repeats its third argument (the \emph{template}) for
each element in the collection, with the \texttt{a:value} tag set to the
element's string each time. The `a' prefix comes from the second argument,
which is a string to prefix each field's tag name with inside the template. 
\indtag{collection item fields!index}
The \texttt{foreach} also sets up an \texttt{a:index} tag giving
the index (starting at zero) of each element. 
There's a more complete explanation of \texttt{foreach} below in section \ref{moreiter}.

\subsection{Accessing elements by index}
\label{collbyindex}
You can also access each element in a collection by using the zero-based index of the
element you want as an argument: \verb,{{mycollection|1}}, would return the
string `second item' in the above example. This means you could access two
collections in parallel, iterating over one collection and using the index
to access the other collection:

\begin{MyVerbatim}
newsdate += 1st January 1990
news += Stuff has happened!

newsdate += 2nd January 1990
news += Oh no, it's happened again!

newsdate += 3rd January 1990
news += And for a third time, events have occurred!

listofitems = [[
   {{foreach|{{page:newsdate}}|a|
        <div class=newsdate>{{a:value}}</div>
        <div class=newsitem>{{page:news|{{a:index}}}}</div>
    }}
   </ul>
]]
\end{MyVerbatim}

You wouldn't actually do anything like that in real code --- if you wanted
to store multiple fields in a collection, you'd actually use a complex
collection.

\clearpage
\section{Complex collections}
\useglosentry{complexcoll}
The collection we created above just has one field: `value.' If you want your collection items
to have multiple fields, you must use a \emph{complex collection.}
To define a complex collection and add the first item, start a line of tag definition language with the `++'
marker followed by the collection name, followed by the field names separated by commas:
\begin{MyVerbatim}
++mycollection:field1,field2,field3
\end{MyVerbatim}
All subsequent lines will now set fields inside the new collection's first item,
rather than creating new tags.
To start creating another item, use `++' followed by the collection name.
You \emph{must exit this mode} by using `++' on a line by itself in order to define new tags. Here's an
example:
\begin{MyVerbatim}
++news:date,text
date=1st January 1990
text=Stuff has happened!

++news
date=2nd January 1990
text=Oh no, it's happened again!

++news
date=3rd January 1990
text=And for a third time, events have occurred!

++
\end{MyVerbatim}
The first line defines the collection and specifies the fields which are available, and creates
the first item. Each `++' line after than adds a new item to the collection. The equals lines
then set fields in the item just created.
At the end, the `++' without a name indicates that we're switching back
into normal mode, where we define tags rather than fields in a collection.

Finally, it's possible to leave fields undefined, in which case they'll
produce errors if you try to expand them. It is, however, possible to
test if a field is defined using the \texttt{iffieldexists} tag.
\indtag{iffieldexists}

You can iterate through a collection and access items within it in exactly the same
way as with simple collections, using \texttt{foreach}. Instead of just
assigning the \texttt{\emph{prefix}:value} field, each field in each item will be assigned 
for each iteration of the loop.

\clearpage
Here's an example of the above news tag system written using complex collections:
\begin{MyVerbatim}
++news:date,text
date=1st January 2010
text=someone somewhere
++news
date=2nd January 2011
text=is having a toffee crisp
++news
date=3rd january 2012
text=[[and now he's finished eating the toffee crisp,
and he's working on a mars bar. Lovely!]]
++

content=[[
    {{foreach|{{page:news}}|a|
        <div class=newsdate>{{a:date}}</div>
        <div class=newsitem>{{a:text}}</div>
    }}
]]
\end{MyVerbatim}

\subsection{Accessing complex collections by index}
\label{collbyindex2}
As we saw above, we can access a simple collection by index by doing something like \verb,{{coll|0}},.
We can also specify the field name so that this can be extended to complex collections:
\begin{MyVerbatim}
{{coll|0|myfield}}
\end{MyVerbatim}
will get the value of \texttt{myfield} in item 0 of the collection.

\section{Collection handles}
\useglosentry{handle}
Whenever SCMS creates a collection, it creates a handle for it:
a string in the form \texttt{coll:\emph{integer}}. When we define a tag as
referring to a collection using the `+=' or `++' notation, we create the collection
and store the collection's handle in the tag. This can then be accessed
using the tag with no arguments. For example, in our news example above, if 
we used the \texttt{page:news} tag with no arguments we would get the string
\texttt{{coll:0}} or something similar in our output --- the handle string.

We do this because the templating system works entirely on text strings,
and therefore tag arguments have to be text strings. However, we sometimes need
to be able to have collections as tag arguments (such as to \texttt{foreach}
above). Substituting a collection handle for an actual collection allows
us to do that. In addition, it lets us have collections inside collections
--- as we'll see later on with trees and menus.

\section{More collection iteration}
\label{moreiter}
\indtag{foreach}
As noted above, we iterate over a collection using the \texttt{foreach}
tag. This takes the collection handle, a tag prefix, and a template as its arguments.
The tag operates by doing the following for each item:
\begin{itemize}
\item For each field in the item, define a tag \texttt{\emph{prefix}:\emph{fieldname}} with the value
of the field.
\indtagsec{collection item fields!index}
\item Define the field \texttt{\emph{prefix}:index} containing the index of the item.
\indtag{collection item fields!handle}
\item Define the field \texttt{\emph{prefix}:handle} containing the node handle of the item\footnote{see Chapter~\ref{trees} for details on node handles, you don't need them for most collections.}.
\item Process the template, expanding all tags within it.
\item Concatenate the result onto the result string (which starts empty).
\end{itemize}
\clearpage

It's possible to nest foreach loops. For example, you can place one collection inside another and iterate
over them like this:

\begin{MyVerbatim}

## Create two simple collections, page:inner1 and page:inner2

inner1+=inner collection 1, item 1
inner1+=inner collection 1, item 2
inner1+=inner collection 1, item 3
inner2+=inner collection 2, item 1
inner2+=inner collection 2, item 2
inner2+=inner collection 2, item 3

## Create a complex collection, page:outer. It has two fields:
## title is the name of an item, collection is a collection handle.

++outer:title,collection
title=Item 1
collection={{page:inner1}}

++outer
title=Item 2
collection={{page:inner2}}

++

## Now iterate through the outer collection, and within that,
## through the collection indicated by the 'collection' field of
## the current item.

content=[[
    <ul>
    {{foreach|{{page:outer}}|outer|
        <li>Outer collection - {{outer:title}}
        <ul>
            {{foreach|{{outer:collection}}|inner|
                <li>{{inner:value}}</li>}}
        </ul>}}
    </ul>
]]
\end{MyVerbatim}

\section{Creating collections from strings}
\label{collsplit}
It's possible to split a string into a collection using the \texttt{split}
tag. This takes two arguments: the string to split, and the delimiter. It
returns the collection handle. Here's
an example:
\begin{MyVerbatim}
{{foreach|{{split|foo/bar/baz|/}}|f|
    <p>{{f:value}}</p>}}
\end{MyVerbatim}
This will print separate paragraphs for the three strings \texttt{foo},
\texttt{bar} and \texttt{baz}.     

\section{Debugging}
It's possible to dump a collection or tree to an HTML table using the \texttt{dumpcoll} tag:
\begin{MyVerbatim}
{{dumpcoll|{{navtree}}}}
\end{MyVerbatim}
\indtag{dumpcoll}
will dump the top level of the navigation tree.

\section{Implementation of collections}
\label{collimp}
\todo{Not yet written, but it's fairly straightforward --- see
\texttt{createlangtree()} in \texttt{langmenu.php} for an example of
how to create a simple list (although it's set up as a tree, for which
see the next chapter) and \texttt{navmenu.php} for a much
more complex example.}
