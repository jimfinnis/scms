%%%%%%%%%%%%%%%%%%%%%%%%%%%%%%%%%%%%%%%%%%%%%%%%%%%%%%%%%%%%%%%%%%%%%%%%%%%%%
%
%                     Copyright 2010 Jim Finnis.
%                           All Rights Reserved
%
%
%  System        : 
%  Module        : 
%  Object Name   : $RCSfile$
%  Revision      : $Revision$
%  Date          : $Date$
%  Author        : $Author$
%  Created By    : Jim Finnis
%  Created       : Sun Jan 3 15:36:51 2010
%  Last Modified : <120725.1014>
%
%  Description 
%
%  Notes
%
%  History
% 
%%%%%%%%%%%%%%%%%%%%%%%%%%%%%%%%%%%%%%%%%%%%%%%%%%%%%%%%%%%%%%%%%%%%%%%%%%%%%
%
% Copyright (c) 2010 Jim Finnis.
% 
% All Rights Reserved.
% 
% This  document  may  not, in  whole  or in  part, be  copied,  photocopied,
% reproduced,  translated,  or  reduced to any  electronic  medium or machine
% readable form without prior written consent from Jim Finnis.
%
%%%%%%%%%%%%%%%%%%%%%%%%%%%%%%%%%%%%%%%%%%%%%%%%%%%%%%%%%%%%%%%%%%%%%%%%%%%%%

\chapter{Page Caching}
\label{cachechap}
SCMS uses two caching strategies:
\begin{itemize}
\useglosentry{servercache}
\item In \emph{server-side caching} SCMS stores each page it generates in a file, and returns the file
if the page is requested again, rather than regenerating using the template engine. This process is very, very
fast, reducing CPU usage; but has no effect on network usage since the same data is sent to the browser.
\useglosentry{clientcache}
\item In \emph{client-side caching} SCMS forces the user's browser to store the page, so that if they 
request it again the system doesn't access the network at all, reducing network usage.
\end{itemize}
Neither of these options is any use for dynamic pages, and client-side caching may cause changes to 
pages to not be immediately reflected in what users see unless they explicitly refresh the page.

\section{Enabling caching}
Both kinds of caching are disabled by default, and can be enabled by modifying the following 
lines in \texttt{site/config.php}:
\begin{MyVerbatim}
define('CACHING',0); // SERVER CACHING MODE
define('BROWSERCACHING',0); // CLIENT CACHING MODE
\end{MyVerbatim}
To enable server-side caching, set \texttt{CACHING} to 1.
To enable client-side caching, set \texttt{BROWSERCACHING} to 1.

\section{Server-side caching}
As mentioned above, SCMS' server side caching involves storing the generated HTML for each page
if it isn't in the cache, and serving that for subsequent requests.

First, a cache key is generated, using an hash
based on the following items in the HTTP request:
\begin{itemize}
\item page name
\item \texttt{page} GET variable
\item \texttt{action} GET variable
\item \texttt{c} GET variable
\item \texttt{p} GET variable
\item \texttt{feed} GET variable
\item \texttt{tagid} GET variable
\item \texttt{pagenumber} GET variable
\end{itemize}
\subsection{Extra cache keys}
You may need to add to this list --- to do this, create a \verb+$cacheitemlist+ array
in your \verb+config.php+ file:
\begin{MyVerbatim}
$cacheitemlist = array('key','anotherkey','etc');
\end{MyVerbatim}

The cache is checked for a file whose name is the cache key. If this doesn't exist, the page contents
are generated as usual, and stored in a file of that name before being returned to the client. If it does
exist, the contents of that file are returned.

\subsection{Clearing the cache}
If you have server-side caching enabled you \emph{must} clear the cache
whenever you change a page to ensure that the updated version of the page is
served. This is done by deleting all the cache files. A script called
\texttt{clearcache} is provided to do this, which must be run from the top
level of your SCMS installation. It will also clear the navigation and
language cached data.

\section{Client-side caching}
Setting client-side caching on will cause the browser to cache pages it receives for up to a day. 
If the user requests the page again, they'll just get the version their browser has in its cache unless
they do a refresh (i.e. use the F5 key in most browsers.) The disadvantage of this is that if you change
a page and clear all your caches, the user isn't guaranteed to see it until a day has passed even if you
turn caching off straight away. The advantage is that your network usage may be reduced.


\section{Cache exceptions}
You may have some pages which are dynamic, and therefore should never be cached by either server
or client. The specifiers\footnote{i.e. the name of the page's file within \texttt{site/pages}, see
Chapter~\ref{chappages}.} of these pages can be added to
the \emph{cache exceptions list} which is in defined in \texttt{config.php}:
\begin{MyVerbatim}
$cache_exceptions = array();
\end{MyVerbatim}
For example, you could change this line to:
\begin{MyVerbatim}
$cache_exceptions = array('news','logs/log','admin','search');
\end{MyVerbatim}
to have those pages never be cached on the server or the client.

