
%%%%%%%%%%%%%%%%%%%%%%%%%%%%%%%%%%%%%%%%%%%%%%%%%%%%%%%%%%%%%%%%%%%%%%%%%%%%%
%
%                     Copyright 2010 Jim Finnis.
%                           All Rights Reserved
%
%
%  System        : 
%  Module        : 
%  Object Name   : $RCSfile$
%  Revision      : $Revision$
%  Date          : $Date$
%  Author        : $Author$
%  Created By    : Jim Finnis
%  Created       : Sun Jan 3 15:35:44 2010
%  Last Modified : <110102.1926>
%
%  Description 
%
%  Notes
%
%  History
% 
%%%%%%%%%%%%%%%%%%%%%%%%%%%%%%%%%%%%%%%%%%%%%%%%%%%%%%%%%%%%%%%%%%%%%%%%%%%%%
%
% Copyright (c) 2010 Jim Finnis.
% 
% All Rights Reserved.
% 
% This  document  may  not, in  whole  or in  part, be  copied,  photocopied,
% reproduced,  translated,  or  reduced to any  electronic  medium or machine
% readable form without prior written consent from Jim Finnis.
%
%%%%%%%%%%%%%%%%%%%%%%%%%%%%%%%%%%%%%%%%%%%%%%%%%%%%%%%%%%%%%%%%%%%%%%%%%%%%%

\chapter{Pages}
\useglosentry{page}
\useglosentry{pagetag}
\label{chappages}
Each page in a site has a single file in the \texttt{site/pages} directory,
written in tag definition language. This file defines the tags used by the
template to render the page's content --- tags which are created with the
\texttt{page:} prefix. You can call this file almost anything you like, provided it's
valid as both a filename and as part of a URL. The page specifier will be the
path of the file relative to \texttt{sites/pages}. One small caveat: the file
cannot be called \texttt{defaults}. This name is reserved for the defaults
file in a page directory, containing tags loaded in by all pages in or below that
directory.

Consider our example template, consisting of the \texttt{main.html} file
described in section~\ref{samplemain}. We see that
the template requires the tag \texttt{page:content} to be defined. We
\indtag{page:template}
also define \texttt{page:template} so that SCMS will know which template to
\indtag{page:name}
use for this page\footnote{If we don't give the template name, the system will use the \texttt{default} template.}, and \texttt{page:name} which is the page's name in
navigation menus (and is also used by the template). We can do this by putting
the following in the file \texttt{site/pages/home}:

\begin{MyVerbatim}
 name=The Home Page
 template=default
 content=[[
   <p>This is a test page, which doesn't really do very
   much at all.</p>
 ]]
\end{MyVerbatim}

To display this page, we'd request the URL \texttt{http://../index.php/home}.
SCMS will then do the following:
\begin{itemize}
 \item Open the page file, \texttt{site/pages/home}.
 \item Define the tags as specified in the file: \texttt{page:template},
  \texttt{page:name} and \texttt{page:content}.
 \item Read the template specified in \texttt{page:template} --- in this case, all the data will come from the \texttt{default} template directory.
 \item Process the template's \texttt{main.html} file, resolving the tags.
 \item Output the concatenated result.
\end{itemize}

\section{Special page tags}
These are page tags to which SCMS ascribes special meaning. All other tags only have the meanings you give them by using
them elsewhere in your templates and pages.
\begin{itemize}
\useglosentry{pagename}
\item \usetag{page:name} is a short string (preferably a single word) used as the name of the page in navigation. Examples
are ``Home Page'' and ``News.''
\item \usetag{page:template} specifies which template we should use. Templates consist of a set of files inside a template directory, and are used to define the overall look of the whole site or a group of pages --- see page~\pageref{templatedirs}, Template Directories, for more details. If no template is specified, the template \texttt{default} will be used.
\end{itemize}

\section{Redirection}
\useglosentry{redir}
\indtag{page:redir}
In some cases, a page redirects to an external site. To do this, create a page file which specifies the
name and the tag \texttt{page:redir}, which gives the external URL. For example,
\begin{MyVerbatim}
 name=BBC News
 redir=http://news.bbc.co.uk/
\end{MyVerbatim}
All other tags will be ignored if you use \texttt{redir}.

\section{Hierarchies}
Hierarchies of pages can be created inside the \texttt{site/pages} directory, with the page specifiers being
the filenames of the pages relative to that directory.
For example, if there's a page in \texttt{site/pages/foo/bar}, the page specifier
will be \texttt{foo/bar} and the URL will be something like \texttt{http://.../index.php/foo/bar}.
Loading these pages is not, however, as simple as just reading that file and defining the tags. First
we have to read and define tags in all the \texttt{defaults} files on the way down to the file.

\subsection{The defaults file}
\useglosentry{pagedefaults}
Before a page file is loaded in, any files called \texttt{defaults} that are in
that file's directory and each directory above it (up to \texttt{site/pages})
are loaded in and processed, in descending order, so that tags defined in
deeper files will be read later and so supercede those in files higher up the
hierarchy. If the page specifier references a directory rather than a file,
the \texttt{defaults} file of that directory is the last file read.

This mechanism provides a way to set a large number of tags which are used in
a lot of (but not all) pages, although ideally this job is left to templates.
One thing it is useful for, perhaps, is setting the \texttt{page:template} tag
for all the pages in a directory. It really comes into its own in hierarchies
--- imagine a pages directory containing the following files and directories
(marked in $italics$):
\begin{center}
% use packages: array
\begin{tabular}{lll}
\texttt{home} & \emph{\texttt{rooms}} & \texttt{contact}
\end{tabular}
\end{center}
Now the \emph{rooms} directory:
\begin{center}
% use packages: array
\begin{tabular}{lll}
\texttt{defaults} & \emph{\texttt{upper}} & \emph{\texttt{lower}}
\end{tabular}
\end{center}
And finally the \emph{upper} directory:
\begin{center}
% use packages: array
\begin{tabular}{lll}
\texttt{room1} & \texttt{room2} & \texttt{room3}
\end{tabular}
\end{center}
Now consider viewing pages at each level:
\begin{itemize}
 \item Opening \texttt{home} : here, we'll try to open \texttt{site/pages/defaults}, but it doesn't exist. We'll just open \texttt{site/pages/home} and use the values in it.
\item Opening \texttt{rooms} : First, we'll try to open \texttt{site/pages/defaults} as before, failing. Then we'll try to open \texttt{site/pages/rooms/defaults} and succeed. We'll leave it there because this is a directory, not a file.
\item Opening \texttt{rooms/upper} : We perform the two steps in the last example, and then try to open \texttt{rooms/upper/defaults}. This fails, so we stop there (since this is a directory).
\item Opening \texttt{rooms/upper/room1} : The same as the previous example --- failing to read \texttt{site/pages/defaults}, reading \texttt{site/pages/rooms/defaults}, failing to read \texttt{site/pages/rooms/upper/defaults}. Finally we read \texttt{site/pages/rooms/upper/room1}.
\end{itemize}

\subsection{Marking defaults files as standalone}
A danger here, though, is that we might end up with valid pages containing no
content: some defaults files might be opened containing a few tags, with no
actual content tags set. In order to avoid this, if a \texttt{defaults} file
contains enough information to make a page by itself, it must contain the
\indtag{page:standalone}
\texttt{page:standalone} tag. It doesn't matter what you set it to. Note also
that it has to be in the \texttt{defaults} file of the directory itself ---
this tag is not inherited from parent \texttt{defaults} files.

Therefore, in the second and third cases above (\texttt{rooms} and
\texttt{rooms/upper}) we we'll get a 404 page not found error unless we put
this tag into \texttt{rooms/defaults} or \texttt{rooms/upper/defaults}
respectively.

Note also that we can't reference the \texttt{defaults} files as pages
directly. For example, we can't use \texttt{rooms/upper/defaults} as a
specifier, even if \texttt{defaults} has a standalone tag. If we want to
reference just that file, we should use \texttt{rooms/upper}.

\section{The page file hierarchy is not the page navigation hierarchy}
Note that the hierarchy of page files in the pages has nothing necessarily to
do with the hierarchy of pages as shown in the navigation menu --- the two may
be similar, but the latter is defined by the \texttt{navigation} file, not the
page directory hierarchy!

