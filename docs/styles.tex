%%%%%%%%%%%%%%%%%%%%%%%%%%%%%%%%%%%%%%%%%%%%%%%%%%%%%%%%%%%%%%%%%%%%%%%%%%%%%
%
%                     Copyright 2010 Jim Finnis.
%                           All Rights Reserved
%
%
%  System        : 
%  Module        : 
%  Object Name   : $RCSfile$
%  Revision      : $Revision$
%  Date          : $Date$
%  Author        : $Author$
%  Created By    : Jim Finnis
%  Created       : Mon Feb 22 17:24:58 2010
%  Last Modified : <100223.1601>
%
%  Description 
%
%  Notes
%
%  History
% 
%%%%%%%%%%%%%%%%%%%%%%%%%%%%%%%%%%%%%%%%%%%%%%%%%%%%%%%%%%%%%%%%%%%%%%%%%%%%%
%
% Copyright (c) 2010 Jim Finnis.
% 
% All Rights Reserved.
% 
% This  document  may  not, in  whole  or in  part, be  copied,  photocopied,
% reproduced,  translated,  or  reduced to any  electronic  medium or machine
% readable form without prior written consent from Jim Finnis.
%
%%%%%%%%%%%%%%%%%%%%%%%%%%%%%%%%%%%%%%%%%%%%%%%%%%%%%%%%%%%%%%%%%%%%%%%%%%%%%

\chapter{Styles}
\label{styles}
\useglosentry{style}
A style is similar to a CSS media type --- it's a code SCMS uses to identify classes of
browser so you can change how you render sites to fit that browser or user preference.
In SCMS you can specify the styles yourself, or rely on SCMS to do it for you.

\section{Getting the style}
The style is determined in one of three ways - an HTTP GET parameter, a user-written PHP hook,
or SCMS' own function.

\subsection{HTTP GET style}
If the HTTP GET parameter \texttt{style} exists, its value is used as the current style.
The other methods aren't used. Therefore the URL
\begin{MyVerbatim}
http://mysite.com/index.php/home?style=print
\end{MyVerbatim}
will always set the \texttt{print} style.

\subsection{The user-written hook}
\useglosentry{hook}
\useglosentry{getst}
If there isn't an HTTP GET parameter called `style', the system looks for the file \texttt{site/getstylename.php}
to determine the style. This is a snippet of PHP in which you can check the values of the
\texttt{\$\_SERVER} array, and set the value of \texttt{\$stylename} accordingly. It should
return `default' as the default style.

As an example, there is a PHP library for mobile platform detection available from \texttt{http://detectmobilebrowsers.mobi/}
which is free for non-commercial use only. Once included, you get a function which returns true 
for mobile browsers. You could implement a mobile style by putting the following into \texttt{site/getstylename.php}:
\begin{MyVerbatim}
include('mobile_device_detect.php');

if(mobile_device_detect(true,true,true,true,true,true,false,false))
   $stylename='mobile';
else
   $stylename='default';
\end{MyVerbatim}

\subsection{The default system}
If there is no GET parameter and no user hook file, SCMS will fall back to the standard method
of determining the style. This will produce the following style strings:
\begin{center}
\begin{longtable}{ll}

iphone & iPhone, iPad and iPod touch \\
textmode & Browsers such as Lynx \\
opera & Opera \\
firefox & Firefox \\
Safari & Safari on a desktop/laptop \\
ie8 & MSIE 8\\
ie7 & MSIE 7\\
ie6 & MSIE 6\\
ie-old & Any older Internet Explorer \\
default & Anything else
\end{longtable}
\end{center}

\clearpage
\section{Using the style: the \texttt{stylemap} file}
\useglosentry{stylemap}
\indtag{style}
The style name is available to SCMS via the \texttt{style} tag,
but it's best to use it to change which template is used. This is done
via the \texttt{site/stylemap} file.

Each line in the file has the form:
\begin{MyVerbatim}[commandchars=+\[\]]
+emph[style/template] => +emph[commands...]
\end{MyVerbatim}
where \texttt{style} is the style to match, template is the template to match
(or * to match any template)
and commands is a list of commands as described below.
When SCMS finds it's rendering a page with the given template and the style
is set to the given style, the commands will be run.

The commands are:
\begin{itemize}
\item \texttt{template \emph{name}} will change the template directory to be used
\item \texttt{mainfile \emph{file.html}} will change the main template file to be used
\end{itemize}

For example, if we have a page which contains the line
\begin{MyVerbatim}
template=biglist
\end{MyVerbatim}
it will normally be rendered using the template in \texttt{site/templates/biglist}.
If we want to change how it is rendered by an iPhone, the style map line
\begin{MyVerbatim}
iphone/biglist => template iphonebiglist
\end{MyVerbatim}
will say that if the current style is \texttt{iphone} and we are
going to use the template directory \texttt{biglist}, we should change the template
to \texttt{iphonebiglist} before we proceed any further and use that
directory instead.

Another example:
\begin{MyVerbatim}
print/* => mainfile print.html
\end{MyVerbatim}
This says that whenever we have the \texttt{print} style, the main file
should be changed from \texttt{main.html} to \texttt{print.html}. This will
happen for every template, so every template directory must contain that
new file.

\section{Linking to another style}
It's possible to generate a link from a page to another or the same page
\indtag{link}
\indtag{url}
in a different style using the \texttt{link} or \texttt{url} tag. They're
similar, but the former generates an entire \texttt{<A HREF>} HTML tag,
while the latter generates just the URL (useful for image links). The \texttt{link} tag
takes the page specifier, the link text and the style; while the \texttt{url} tag
takes just the specifier and the style:
\begin{MyVerbatim}
{{link|home|Home page in Print style|print}}
{{url|home|print}}
\end{MyVerbatim}
both create links to the home page in a print-friendly version (provided you've
set up a style map and written a printer-friendly template!)

If you want to generate links to the current page in another style, you use the \texttt{spec} tag 
to fetch the current page's specifier:
\begin{MyVerbatim}
{{link|{{spec}}|printer friendly|print}}
{{url|{{spec}}|print}}
\end{MyVerbatim}
You can leave out the style if you like, in which case they become handy
tags for generating links around your site.

