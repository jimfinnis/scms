%%%%%%%%%%%%%%%%%%%%%%%%%%%%%%%%%%%%%%%%%%%%%%%%%%%%%%%%%%%%%%%%%%%%%%%%%%%%%
%
%                     Copyright 2010 Jim Finnis.
%                           All Rights Reserved
%
%
%  System        : 
%  Module        : 
%  Object Name   : $RCSfile$
%  Revision      : $Revision$
%  Date          : $Date$
%  Author        : $Author$
%  Created By    : Jim Finnis
%  Created       : Sun Jan 3 15:36:10 2010
%  Last Modified : <131117.1833>
%
%  Description 
%
%  Notes
%
%  History
% 
%%%%%%%%%%%%%%%%%%%%%%%%%%%%%%%%%%%%%%%%%%%%%%%%%%%%%%%%%%%%%%%%%%%%%%%%%%%%%
%
% Copyright (c) 2010 Jim Finnis.
% 
% All Rights Reserved.
% 
% This  document  may  not, in  whole  or in  part, be  copied,  photocopied,
% reproduced,  translated,  or  reduced to any  electronic  medium or machine
% readable form without prior written consent from Jim Finnis.
%
%%%%%%%%%%%%%%%%%%%%%%%%%%%%%%%%%%%%%%%%%%%%%%%%%%%%%%%%%%%%%%%%%%%%%%%%%%%%%

\chapter{Modules}
\label{moduleschap}
Modules let you create extensions to SCMS using PHP. They can do two things:
\begin{itemize}
\item Register new tags, which can be strings, functions or collections.
\item Provide PHP code for responding to HTTP GET and POST requests, which 
run code before the rest of the page is generated.
\end{itemize}
Naturally, you'll need to be fairly confident in PHP!

\section{The modules directory}
Modules reside in the \texttt{site/modules} directory, which you'll
need to create if you want to use modules. Each module
has its own directory named after the module. There are several modules
provided in the main SCMS directory which you should link into your
site's modules directory as you require them.

\section{Loading modules}
To find which modules to load, SCMS checks both the \texttt{page:modules} and
\texttt{template:modules} tags and loads the modules specified in
both. Each tag should be defined as a comma-separated list of dispositions and modules. The disposition
is either `post, `get' or `all'. If the disposition is `post' or `get' , the module is only
loaded if it is requested via the POST or GET interface by setting the request parameter `mod' to the module name.
Modules marked 'all' are always loaded for the page or template.
Each module file contains the definition of a module object, which it adds to
Modules object.

\noindent An example:
\begin{MyVerbatim}
all:foo,post:bar
\end{MyVerbatim}
This will always load the `foo' module, but only load the `bar` module if it is specifically requested by a POST setting
\verb+$_POST['mod']+ to `bar'.

\section{Creating a module}
To create a module, you need to create a directory for it and 
create a \texttt{main.php} file inside it. You can --- and should, for
large modules --- create other files and \texttt{require} them if you
wish. The main file should contain either or both:
\begin{itemize}
\item code to register tags,
\item a subclass of \texttt{Module} which responds to POST and GET, and
code to instantiate and register it with the module system.
\end{itemize}
If you're just going to register some new tags you don't need a Module object.

\section{Registering tags}
Here's a very simple module, which we'll call \texttt{ShowGenDate}. We
create a directory of that name in the site modules directory, and 
write the \texttt{main.php} file:

\begin{MyVerbatim}
<?php

$date = date(DATE_COOKIE);
Template::regstring('ShowGenDate:GenDate',$date,"date of page build");

\end{MyVerbatim}

And that's all: we have registered the tag \texttt{ShowGenDate:GenDate} to hold the
string containing the time at which the code was run --- i.e. when the
page was last generated. The third argument is a description of the tag.
Now, we can use this tag in any page:
\begin{MyVerbatim}
content=[[
    Hello. Page content last built on
            {{ShowGenDate:GenDate}}
]]
\end{MyVerbatim}
Note the important convention of naming the tags defined by a module as 
\begin{MyVerbatim}
    ModuleName:TagName
\end{MyVerbatim}
This prevents namespace clashes, but it is only a convention! When you register tags
in a module, you're using exactly the same functions the core system uses. In fact, 
reading through \texttt{core/functpls.php} can be useful for figuring out how to write
tags --- it defines the majority of the standard tags.

\section{Function tags}
Creating tags which actually do something is very easy too. Let's create a tag
which just concatenates two strings together:
\begin{MyVerbatim}
function MyModule_concat($t)
{
  $t->assertargc(2);
  $a = $t->argv[0]);
  $b = $t->argv[1]);
  return $a.$b;
}
Template::regfunc('MyModule:concat','join two strings','str1|str2',
    MyModule_concat);
\end{MyVerbatim}
This is how the function works:
\begin{itemize}
\item A \texttt{TagInText} structure is passed in, containing the arguments and the tag name. You can see 
more details of this in \texttt{core/template.php}. 
\item The argument count is checked, and the page render will abort if it isn't equal to 2. You can use the
\texttt{argc} member directly if you want to do something smarter, such as variable arguments.
\item Arguments 0 and 1 of the tag are read in.
\item The arguments are joined together and returned.
\end{itemize}
If we create a \texttt{MyConcat} module and add the above code to its \texttt{main.php} 
we can now do
\begin{MyVerbatim}
{{MyModule:concat|foo|bar}}
\end{MyVerbatim}
and get the correct answer
\begin{MyVerbatim}
foobar
\end{MyVerbatim}
\subsection{Processing tag arguments}
However, there's a major problem:
\begin{MyVerbatim}
{{MyModule:concat|{{langcode}}|{{langencoding}}}}
\end{MyVerbatim}
will give us
\begin{MyVerbatim}
{{langcode}}{{langencoding}}
\end{MyVerbatim}
instead of the expected
\begin{MyVerbatim}
en_USUTF-1
\end{MyVerbatim}
What's happened is that the arguments to our tag function have not, themselves, been processed by the templating
system to expand any tags within them. Most tag functions (but not all) do this, so we'll modify our code to do it:
\begin{MyVerbatim}
function MyModule_concat($t)
{
  $t->assertargc(2);
  $a = Template::process($t->argv[0]));
  $b = Template::process($t->argv[1]));
  return $a.$b;
}
Template::regfunc('MyModule:concat','join two strings','str1|str2',
    MyModule_concat);
\end{MyVerbatim}
Now it'll work.

One of the function tags which don't process its arguments is \texttt{treeselnode} (and its siblings.) Can you see why?

\todo{\texttt{Module} subclasses which can respond to \texttt{GET} and
\texttt{POST}}
\todo{special modules with no main (c.f. captchas)}
\todo{predefined modules}
