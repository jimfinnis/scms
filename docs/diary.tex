%%%%%%%%%%%%%%%%%%%%%%%%%%%%%%%%%%%%%%%%%%%%%%%%%%%%%%%%%%%%%%%%%%%%%%%%%%%%%
%
%                     Copyright 2009 Jim Finnis.
%                           All Rights Reserved
%
%
%  System        : 
%  Module        : 
%  Object Name   : $RCSfile$
%  Revision      : $Revision$
%  Date          : $Date$
%  Author        : $Author$
%  Created By    : Jim Finnis
%  Created       : Wed Dec 9 17:42:32 2009
%  Last Modified : <091216.2002>
%
%  Description 
%
%  Notes
%
%  History
% 
%%%%%%%%%%%%%%%%%%%%%%%%%%%%%%%%%%%%%%%%%%%%%%%%%%%%%%%%%%%%%%%%%%%%%%%%%%%%%
%
% Copyright (c) 2009 Jim Finnis.
% 
% All Rights Reserved.
% 
% This  document  may  not, in  whole  or in  part, be  copied,  photocopied,
% reproduced,  translated,  or  reduced to any  electronic  medium or machine
% readable form without prior written consent from Jim Finnis.
%
%%%%%%%%%%%%%%%%%%%%%%%%%%%%%%%%%%%%%%%%%%%%%%%%%%%%%%%%%%%%%%%%%%%%%%%%%%%%%

\documentclass[10pt,twoside,openright]{scrbook}
\usepackage[hmargin={0.75in,0.5in},twoside]{geometry}
\geometry{papersize={6in,9in}}
\special{papersize=6in,9in}
\usepackage{tocloft}
\setlength\cftsecnumwidth{3em}

\newcommand{\HRule}{\noindent\rule{\linewidth}{0.1mm}\\}
\newcommand{\blankline}{\vspace{1ex}}
\newcommand{\subscript}[1]{\ensuremath{_{\tiny\textrm{#1}}}}
\newcommand{\superscript}[1]{\ensuremath{^{\tiny\textrm{#1}}}}
\newcommand{\jref}[2]{\textsc{#1}\subscript{\pageref{#2}}}
\newcommand{\orn}{\begin{center}$$\bullet$$\end{center}}

\title{SCMS notes}
\date{December 2009-}
\begin{document}
\maketitle
\tableofcontents

\section{Directories and Permissions}
\begin{itemize}
\item \texttt{core} contains core PHP files for the system which are the same at all sites.
\item \texttt{site} contains template, page and content files for the site.
\item \texttt{tmp} contains temporary cached data for the system and should be writable
\item \texttt{tmp/cache} contains temporary cached data for pages and should be writable
\end{itemize}

\section{Languages}
The language system works on a cached basis. For each language the site
supports, a \texttt{.nls.php} file should be copied into the \texttt{site/languages}
directory. When a Rebuild is done (see Rebuilding), the cached language data file
\texttt{tmp/langs} will be built. This avoids reading all the language files
each time the system starts up.

The language data thus built --- either from the cache or directly from
the NLS files --- consists of \texttt{LangEnt} structures, each of which
has a language code (an ISO code), an endonym (the name of the language
for itself), an exonym (the name of the language in English) and an
encoding (such as UTF-8). There are calls to set and retrieve the language
data.

\section{Rebuilding}
Whenever the language data or cached navigation data is changed it should
be rebuilt. This is done by either deleting the cached versions of the files
(i.e. everything in tmp\footnote{but leave an empty cache directory}) or by 
calling the \texttt{rebuild\_all()} function in the core.

\section{Tag definition files}
These are read by the \texttt{Template} singleton, which manages the templating
system. They provide a way to define tags which the templater replaces with
text.

The function \texttt{Template::readtagdeffile() takes a namespace, a filename and a directory. It first attempts to read a file
from the directory, and then a file from the language directory for the
current language within that directory. If a file exists in both directories,
entries in the language-specific file will override those in the first file.
The namespace string is prepended to the name of any tag defined by
these means.
The file is of the form:
\begin{verbatim}
# comment if start of line is a hash, blank lines ignored

name=value
name2=another value
name3=[[multiline
value]]

#delimiter changes

!delim {{,}}
name4={{multiline
value}}
\end{verbatim}

\section{URLs}
URLs passed to SCMS are in the form
\begin{verbatim}
index.php/foo/bar/baz
\end{verbatim}
The part after the .php is the name of a \emph{page file}.
This is a tag definition file which is in the page directory.
The hierarchy of the names does not necessarily
match that of the navigation structure! For example in this structure:
\begin{verbatim}
index.php/home
index.php/accommodation/room1
index.php/staff/peter.smith
\end{verbatim}
the files would have to be stored in the pages directory in subdirectories
as indicated:
\begin{verbatim}
./site/pages/home
./site/pages/accommodation/room1
./site/pages/staff/peter.smith
\end{verbatim}
but the navigation file (q.v.) might describe a quite different hierarchy
\footnote{
We can't enforce the hierarchies to be the same, because we might have
a situation where (to follow the above example) there is an `accommodation'
page and also an `accommodation' directory. Something can't be both a file
and a directory, and to make this work would involve some complications, so
I decided not to.}.

If the first element is a valid language code, it is stripped off and
the language is set to that code. So 
\begin{verbatim}
index.php/en_US/home
index.php/en_US/accommodation/room1
index.php/en_US/staff/peter.smith
\end{verbatim}
are all valid and equivalent to the above example, assuming that the
default language is US English. Beware of trying to create pages
at the root level that are the name of a language or language alias!
\end{document}

\section{Templates}
Templates are stores as subdirectories of the \texttt{templates} directory
in the site directory. Each template has at least three files in it,
which may also be duplicated in a language subfile. The first two files,
\texttt{head} and \texttt{body}, are files containing HTML templates
which are concatenated together and processed by the template system
to build the page content. The remaining files are all suffixed by
\texttt{.tag} and are tag definition files containing tags useful for constructing
the page. These will all be build with the \texttt{template:} prefix, and will
all be read automatically.

Language-wise, the tags file works the way all tag definition files work:
both files are read in succession and tags in the language-specific
file override those in the base file. The body and head files work
differently: if a language-specific file exists, that is used; otherwise
the base file is used.

\subsection{Tags}
The templater works by processing the text to find tags, which are 
pieces of text between the delimiters {{ and }}. It then finds the string
value of the tag (which may be just a literal string or may require calling
a function) and replaces the tag and its delimiters with the resulting
text. If there are any tags remaining in the text, it's processed again and
again until all the tags are gone. That way, tags values can contain other
tags.

Tags can take arguments, which are specified by 
putting them after the tag name, separated by vertical bar characters.
For example
\begin{verbatim}
{{cmpn|eq|{{level}}|4|level 4|not level 4}}
\end{verbatim}
uses the \texttt{cmpn} (compare numeric) tag to test the value of the 
tag \texttt{level} (after processing) to see if it equals the number 4.
If so, ``level 4'' is output, otherwise ``not level 4'' is output.

Many of these tags are used with arguments to set values inside the 
system, rather than to output values. See the menu system for examples.
Some tags are only available in a certain context -- again, the menu system
has tags which are only defined when the system is rendering a menu.

The user can even define their own tags, using the \texttt{set} tag.

Finally, note that if a tag used in a template is not defined, an
exception will usually result. However, if the tag name is prefixed
with @, the exception will not occur and the tag will resolve to the
empty string.

\section{Menus}
The problem with menus, if we want to generalise the concept to cover
all kinds of menu from the navigation menu to language selection, is
how on earth does the system know which item is the current item?
And how do we determine the URIs for the links? In the case of non-navigation links,
just a query item will change (such as ``?mode=edit'' or something like that),
but the language menus are very odd in that they change a top-level part
of the URI but not the lower part.

For example, the language menu might have two options: English and Welsh.
Here, pressing the ``Cymraeg'' button on any English page will change the
URI to replace the existing language string with \texttt{cy\_CY}, or
insert a new one if there isn't one, while keeping the rest of the URI
the same.

Therefore, the menu system must, to build the items, perform a
transformation on the current URI before inserting the new element.
Effectively it's a two part process: firstly, analyse the current URI
and put some kind of tag, say XXXX, where each menu should put something
different. Then each menu item should replace the tag with its own string.
Only the top-level menu needs to know about the transformation and tag-building,
since that's the same for all items. Each menu item of course has its own
string to replace.

Concrete examples. The language menu's transformation is simple. If the top
level of the URI, immediately after the first slash, is a language code, replace it
with the tag. Otherwise, add a new level after that first slash, and make
that the tag. So:
\begin{itemize}
\item \texttt{http://foo.com/en\_US/index.php} becomes \texttt{http://foo.com/XXXX/index.php}
\item \texttt{http://foo.com/fr\_FR/bar/zog.php} becomes \texttt{http://foo.com/XXXX/bar/zog.php}
\item \texttt{http://foo.com/bar.php} becomes \texttt{http://foo.com/XXXX/bar.php}
\end{itemize}

The way to to do this is:
\begin{itemize}
\item A \emph{menu item} consists of a string, a value and a optional
an extra datum. The string is a the string to be displayed as the item's
text, the value is the string used by the menu code to build the new URI,
and the optional datum is exactly that --- one possibility would be the
name of an image filename to use for that item. The value may be a string,
or it may also be an array or more menu items: a submenu. If the item
represents a submenu, there's also a field referencing the selected item
in the submenu.
\item A \emph{menu} is an abstract data type with an array of menu items
and a method the system uses (in response to a template) to build each
menu item's URI, given the current page's URI (as a handy set of data, all
nicely split up). There's also a field reflecting the selected item, and
a method to initialise the menu --- which mainly consists of working out
what the selected item is from the current URI!
\item Finally, output is done using the template tags, which get a bit
complex.
\end{itemize}

\subsection{Menu tags}
\begin{itemize}
\item \texttt{menuprefix} is the tag used to set which string is used to
prefix subsequently rendered menus or submenus. It takes two arguments,
the level of menu for which this prefix applies, and the string to use.
The level is an integer, such as 0 for the top level menu; or the word
`default' meaning this is the string to use for cases where no explicit
level string is provided.
\item \texttt{menusuffix} is the tag used to set which string is used to
suffix subsequently rendered menus or submenus. It takes the same
arguments as \texttt{menuprefix}.
\item \texttt{menuselitem} is the tag used to set the template used for 
rendering a selected menu item. Selected menu items are those which match
the current URL or are part of a hierarchy which matches it. It takes the same
arguments as \texttt{menuprefix}.
\item \texttt{menuunselitem} is the tag used to set the template used for 
rendering an unselected menu item It takes the same
arguments as \texttt{menuprefix}.. 
\item \texttt{rendermenu} is the tag used to render the menu into the
template, once the four templates above have been set. It takes a single
argument, the name of the menu as registered in the global menu map
(\texttt{GlobalData::$menus}).
\end{itemize}

\subsection{Menu rendering}
Rendering is a recursive function. To render a menu, the system
\begin{itemize}
\item Outputs the prefix tag
\item For each item:
\begin{itemize}
\item Sets the item tags to the correct values for that item (see below) so they can be used in the menuselitem
and menuunselitem templates
\item If the item is a submenu, calls the entire render function recursively
to render the submenu and store it in the \texttt{menuitemsubmenutext} tag, where
it can be used by the  menuselitem and menuunselitem templates
\item Outputs the value of either the \texttt{menuselitem} or \texttt{menuunselitem} tag
depending on whether the item is selected or not
\end{itemize}
\item Outputs the suffix tag

\subsection{Menu item tags}
The following tags are defined as the renderer iterates the menu items:
\begin{itemize}
\item \texttt{menuitemlevel} is the current level of the menu -- 0 for the top level menu, 1 for the submenus below that and so on.
\item \texttt{menuitemishome} is true if the current item is the first page (item 0 at the root)
\item \texttt{menuitemindex} is the index of the item within its menu
\item \texttt{menuitemstring} is the string value of the item, typically the on-screen name
\item \texttt{menuitemvalue} is the value of that item, typically the value that forms the URL
\item \texttt{menuitemurl} is the actual URL of the item, obtained by calling \texttt{getURIforitem()} and generally built from the value and the current URL
\item \texttt{menuitemextradata} is the value of the optional extra data field
\item \texttt{menuitemissubmenu} is nonzero if the item is a submenu
\item \texttt{menuitemisselected} is nonzero if the item is selected, or in the navigation path (i.e. it has a selected child)
\item \texttt{menuitemisselleaf} is nonzero if the item is the selected leaf - i.e. the page we're currently viewing
\item \texttt{menuitemsubmenutext} is the string produced by rendering the submenu for this item, or an empty string if it's not a submenu.
\end{itemize}
We also have to do separators and custom items...
