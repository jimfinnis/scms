%%%%%%%%%%%%%%%%%%%%%%%%%%%%%%%%%%%%%%%%%%%%%%%%%%%%%%%%%%%%%%%%%%%%%%%%%%%%%
%
%                     Copyright 2010 Jim Finnis.
%                           All Rights Reserved
%
%
%  System        : 
%  Module        : 
%  Object Name   : $RCSfile$
%  Revision      : $Revision$
%  Date          : $Date$
%  Author        : $Author$
%  Created By    : Jim Finnis
%  Created       : Sun Jan 3 15:36:27 2010
%  Last Modified : <110102.2009>
%
%  Description 
%
%  Notes
%
%  History
% 
%%%%%%%%%%%%%%%%%%%%%%%%%%%%%%%%%%%%%%%%%%%%%%%%%%%%%%%%%%%%%%%%%%%%%%%%%%%%%
%
% Copyright (c) 2010 Jim Finnis.
% 
% All Rights Reserved.
% 
% This  document  may  not, in  whole  or in  part, be  copied,  photocopied,
% reproduced,  translated,  or  reduced to any  electronic  medium or machine
% readable form without prior written consent from Jim Finnis.
%
%%%%%%%%%%%%%%%%%%%%%%%%%%%%%%%%%%%%%%%%%%%%%%%%%%%%%%%%%%%%%%%%%%%%%%%%%%%%%

\chapter{Other Tags}

This chapter deals with tags which haven't been mentioned elsewhere, but which are often useful.

\section{Numeric operations}
\indtag{add}
\indtag{sub}
\indtag{mul}
\indtag{div}
\indtag{int}
\indtag{mod}
Tags are available for performing simple arithmetic. If we
call the arguments $a$, $b$ and so on, and assume they are strings
which evaluate to integers when passed through the template system, the
operations are:
\begin{center}
\begin{tabular}{ll}
\textbf{Tag} & \textbf{Output} \\
\hline
\texttt{add|$a$|$b$} & $a+b$ \\
\texttt{sub|$a$|$b$} & $a-b$ \\
\texttt{mul|$a$|$b$} & $a*b$ \\
\texttt{div|$a$|$b$} & $a/b$ \\
\texttt{mod|$a$|$b$} & $a$ mod $b$ \\
\texttt{int|$a$} & $a$ as integer, or zero if not parseable\\
\end{tabular}
\end{center}

\section{Comparisons and conditions}
\indtagsec{streq}
\indtag{strieq}
\indtag{cmpn}
\indtag{if}
\indtag{ifnotempty}
\indtag{iftagexists}
\indtag{iftagtrue}
\indtag{useifexists}
The following tags are available for performing numeric and string
comparisons, and for conditional code. They all have the same basic
form: some arguments defining the condition, then two arguments for the
outcome --- the first is used if the condition is true, the second
if the condition is false\footnote{See page~\pageref{strequselab} for 
examples with \texttt{streq}.}. Here are the available conditionals:
\begin{center}
\begin{tabular}{lr}
\textbf{Tag} & \textbf{Result} \\
\hline
\texttt{streq|$a$|$b$|$c$|$d$} & $c$ if $a$=$b$, otherwise $d$ \\
\texttt{strieq|$a$|$b$|$c$|$d$} & $c$ if lowercase($a$)=lowercase($b$), otherwise $d$\\
\texttt{cmpn|eq|$a$|$b$|$c$|$d$} & $c$ if int($a$)=int($b$), otherwise $d$ \\
\texttt{cmpn|neq|$a$|$b$|$c$|$d$} & $c$ if int($a$)$\neq$int($b$), otherwise $d$ \\
\texttt{cmpn|gt|$a$|$b$|$c$|$d$} & $c$ if int($a$)$>$int($b$), otherwise $d$ \\
\texttt{cmpn|lt|$a$|$b$|$c$|$d$} & $c$ if int($a$)$<$int($b$), otherwise $d$ \\
\texttt{cmpn|gte|$a$|$b$|$c$|$d$} & $c$ if int($a$)$\geq$int($b$), otherwise $d$ \\
\texttt{cmpn|lte|$a$|$b$|$c$|$d$} & $c$ if int($a$)$\leq$int($b$), otherwise $d$ \\
\texttt{if|$a$|$b$|$c$} & $b$ if $a$ is a non-zero integer, otherwise $c$ \\
\texttt{ifnotempty|$a$|$b$|$c$} & $b$ if $a$ is not an empty string, otherwise $c$ \\
\texttt{iftagexists|$a$|$b$|$c$} & $b$ if $a$ is a valid tag, otherwise $c$ \\
\texttt{iftagtrue|$a$|$b$|$c$} & $b$ if value of tag $a$ is a non-zero integer, otherwise $c$ \\
\texttt{useifexists|$a$|$b$} & value of $a$ if $a$ is a valid tag, otherwise $b$ \\
\texttt{contains|$a$|$b$|$c$|$d$} & $c$ if string $a$ contains $b$, else $d$ \\
\end{tabular}
\end{center}

\noindent Here's an example.
\begin{MyVerbatim}
{{streq|foo|bar|equal|notequal}}
\end{MyVerbatim}
will display \texttt{notequal} because the first two strings are not
equal.
In all cases, all the arguments may contain tags. In the case of the outcomes,
only the outcome actually used will be expanded. That means you can
do things like:
\begin{MyVerbatim}
{{streq|{{httpgetparam|wibble}}|yes|{{loadpage|foo|page}}|}}
\end{MyVerbatim}
This will cause the page \texttt{foo} to be loaded, all its definitions
overloading the current definitions in the page\footnote{because we're
using the \texttt{page} prefix, which is the default prefix for page 
tags.}, but only if the HTTP GET parameter `wibble' has the value \texttt{yes}.

\clearpage
\section{Logical operations}
\indtag{and}
\indtag{or}
\indtag{not}
Typically for use with the \texttt{if} tag above, the following logical
operations are available:
\begin{center}
\begin{tabular}{ll}
\textbf{Tag} & \textbf{Result} \\
\hline
\texttt{and|$a$|$b$} & $a \wedge b$ \\
\texttt{or|$a$|$b$} & $a \vee b$ \\
\texttt{not|$a$} & $\neg a$ \\
\end{tabular}
\end{center}
\subsection{Switch}
\indtag{switch}
The \texttt{switch} tag is SCMS' equivalent to other languages' \emph{switch} or \emph{case} construct.
It uses a string comparison between a test string and several candidates, each of which is paired with
an output string. If the test string matches a candidate, the corresponding output string is expanded
and returned. Here's the syntax:
\begin{MyVerbatim}[commandchars=+\[\]]
{{switch|+emph[teststring]|
    +emph[candidate1]|+emph[output1]
    +emph[candidate2]|+emph[output2]
    +emph[candidate3]|+emph[output3]
    default|+emph[defaultoutput]}}
\end{MyVerbatim}
The special condition string \texttt{default} matches anything, and is used to provide the default case.
It must be the last item. An example of switch usage is on page~\pageref{invisitems}, in the section
dealing with invisible items in the navigation tree.

\section{Numeric for loops}
\indtag{for}
Numeric for loops, like those in BASIC, are done using \texttt{for} tag. They have the syntax:
\begin{MyVerbatim}[commandchars=+\[\]]
{{for|+emph[counter tag name]|+emph[start]|+emph[end]|
    +emph[template]}}
\end{MyVerbatim}
SCMS will set the counter tag to the index and output the processed template for every value of the index
between $start$ and $end$. An example:
\begin{MyVerbatim}
<ul>
{{for|i|1|100|
    <li>{{i}}</li>}}
</ul>    
\end{MyVerbatim}
will produce a bulleted list of all the numbers between 1 and 100.

\section{Text operations}
\subsection{Trimming whitespace}
\indtag{trim}
\indtag{settrim}
The \texttt{trim} tag is used to trim any whitespace from around a string:
\begin{MyVerbatim}
<pre>Hello {{trim|    World   }}!</pre>
\end{MyVerbatim}
will print
\begin{MyVerbatim}
Hello World!
\end{MyVerbatim}
Naturally whitespace is effectively trimmed automatically in HTML text,
so if the \texttt{pre} HTML tag wasn't in the above example you would
see the above output even if the \texttt{trim} tag wasn't there.
Therefore, you might consider turning trimming on all the time for 
the output of all tag processing. To do this, use the  \texttt{settrim}
tag with a non-zero numeric argument. This will cause the PHP \texttt{trim}
function to be called on all values returned from tags. Turn it off again
by using \texttt{settrim} with zero as the argument.

\subsection{Splitting text}
\indtag{split}
The \texttt{splitstring} tag allows you to split a text into several substrings using a delimiter, 
put the substrings into different templates, and concatenate the results together. It's not to be confused with \texttt{split}, which also splits
a string, but puts the results into a collection --- see page~\pageref{collsplit}
for more details.)
The syntax is:
\begin{MyVerbatim}[commandchars=+\[\]]
{{splitstring|+emph[mode]|+emph[delimiter]|+emph[string to split]|
    +emph[template1]|
    +emph[template2]|
    +emph[template3]|...}}
\end{MyVerbatim}
This is probably best expressed with an example:
\begin{MyVerbatim}
{{splitstring|wrap| |foo bar baz blibble|
    {{splitvalue}} is the first string|
    {{splitvalue}} is the second string|
    {{splitvalue}} is the third string}}
\end{MyVerbatim}
will output
\begin{MyVerbatim}
foo is the first string bar is the second string baz is the third string
blibble is the first string
\end{MyVerbatim}
For each substring, the tag \texttt{splitvalue} is set to the string and the tag \texttt{splitindex}
is set to the (zero-based) index. The next template is picked from the list of templates and expanded
with those tag values, and concatenated onto the output.

The mode determines what happens when the system runs out of templates for the substrings:
\begin{itemize}
\item In \textbf{wrap} mode, the system cycles back to the first template. This is shown in the example
above, when the `blibble' substring is output with the first template.
\item In \textbf{clip} mode, the system uses the last template for all subsequent items. This would
have resulted in
\begin{MyVerbatim}
foo is the first string bar is the second string
baz is the third string blibble is the third string
\end{MyVerbatim}
\item In \textbf{stop} mode, the system will not use any subsequent items. This would have resulted in
the entry for `blibble' not being added at all. This is the default mode, so any mode strings other
than `wrap' or 'stop' will cause this behaviour.
\end{itemize}
We can use \texttt{splitstring} to perform the same action on all substrings:
\begin{MyVerbatim}
<ul>{{splitstring|clip| |My dog's got no nose|
    <li>{{splitindex}}: {{splitvalue}}</li>}}</ul>
\end{MyVerbatim}
will result in a bulleted list of the words in the phrase ``My dog's got no nose'' being output,
each word being numbered from zero. Note that we cannot currently split on regular expressions.
Using alternate templates is also easy:
\begin{MyVerbatim}
{{splitstring|wrap| |My dog's got no nose|
    <b>{{splitvalue}} </b>|
    {{splitvalue}} }}
\end{MyVerbatim}
will result in every other word being in bold.

\subsection{Encoding characters into HTML entities}
The common task of turning special characters into HTML entities can be done using the \texttt{htmlentities} tag. For example,
\begin{MyVerbatim}
{{htmlentities|"&<>}}
\end{MyVerbatim}
becomes
\begin{MyVerbatim}
&quot;&amp;&lt;&gt;
\end{MyVerbatim}


\section{Tag manipulation}
\subsection{The registry stack}
\indtag{push}
\indtag{pop}
All tag values are kept in the \emph{tag registry}. You can save the state
of the registry --- and therefore the values of all tags --- on to a stack
by using the \texttt{push} tag, and then restore them by using the
\texttt{pop} tag.

\subsection{Loading tags from files}
\indtag{load}
\indtag{loadpage}
There are two tags provided for doing this. The first, \texttt{loadpage},
is used to load a page file (i.e. a tag definition file in the \texttt{site/pages} 
directory.) The syntax is:
\begin{MyVerbatim}[commandchars=+\[\]]
{{loadpage|+emph[pagespec]|+emph[prefix]}}
\end{MyVerbatim}
The arguments are straightforward: $pagespec$ is the page specification
of the file to be loaded, and $prefix$ is the prefix to be used in defining
the tags, instead of the usual \texttt{page} used by the default page loader.

Note, however, that any \texttt{defaults} file will be loaded too: this is a full
page load. This can cause odd behaviour. For example, if you define collections in
your defaults page, they might end up being loaded twice, resulting in a repeated set of items
being added\footnote{It seems like an odd case, but I've just had it happen to me.}.

A full example of this tag is given on page~\pageref{loadpageexample}.

If you want to load tags from an arbitrary file, and not a page file,
you can use the \texttt{load} tag:
\begin{MyVerbatim}[commandchars=+\[\]]
{{load|+emph[prefix]|+emph[dir]|+emph[file]}}
\end{MyVerbatim}
This will load a file from the given directory (relative to the \texttt{site} directory) into the template registry,
prefixing all tags with the given prefix.

\subsection{Debugging tag file loads}
The \texttt{debugloads} tag sets whether tag file load debugging is on. For example,
\begin{MyVerbatim}
{{debugloads|1}}
\end{MyVerbatim}
will turn it on. When it's on, a \texttt{<pre>} block will be written out to the page
for every load, indicating the file loaded and the type of load.

\section{URL and HTTP manipulation}
\subsection{Retrieving HTTP variables}
This is done using the tags \texttt{httpgetparam} and \texttt{httppostparam},
which return GET and POST parameters respectively. They each take a single
argument: the name of the required parameter.

\subsection{URL construction and analysis}
Several tags exist which help in analysing the URL by which the page
was referenced, and to help to construct other URLs:
\begin{itemize}
\indtag{root}
\item \texttt{root} is a URL pointing to the root directory --- the directory
which contains \texttt{site}, \texttt{core} etc. This is useful for
generating URLs to assets kept in subdirectories outside the normal
SCMS tree.
\item \texttt{imgroot} is a URL pointing to the \texttt{site/images}
directory, so that commonly used images can be accessed by referring to them
as (for example):
\indtag{imgroot}
\begin{MyVerbatim}[commandchars=+\[\]]
<img src="{{imgroot}}image.jpg"/>
\end{MyVerbatim}
\indtag{thispage}
\item \texttt{thispage} returns a URL which references the current page.
\item \texttt{url|$page$|$style$} returns a URL for a page, given the page
specifier
and optionally the style. Useful
for links within the site.
\item \texttt{link|$page$|$text$|$style$} creates a link to a page, given the
page specifier, the link text, and optionally the style.
\item \texttt{defaultpage} is the page specification for the home page (as
given in \texttt{config.php}.) The URL for the home page can be
specification.
\end{itemize}

\section{Changing the default time zone}
It may be necessary to change the default time zone for SCMS. This is done
by using the \texttt{settimezone} tag with a time zone string:
\begin{MyVerbatim}[commandchars=+\[\]]
{{settimezone|Europe/London}}
\end{MyVerbatim}
SCMS uses PHP's \texttt{date\_default\_timezone\_set()} for this
functionality --- see the documentation for that function to find
out what the valid time zone strings are.
