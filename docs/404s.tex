%%%%%%%%%%%%%%%%%%%%%%%%%%%%%%%%%%%%%%%%%%%%%%%%%%%%%%%%%%%%%%%%%%%%%%%%%%%%%
%
%                     Copyright 2010 Jim Finnis.
%                           All Rights Reserved
%
%
%  System        : 
%  Module        : 
%  Object Name   : $RCSfile$
%  Revision      : $Revision$
%  Date          : $Date$
%  Author        : $Author$
%  Created By    : Jim Finnis
%  Created       : Sun Jan 3 15:36:39 2010
%  Last Modified : <110102.1448>
%
%  Description 
%
%  Notes
%
%  History
% 
%%%%%%%%%%%%%%%%%%%%%%%%%%%%%%%%%%%%%%%%%%%%%%%%%%%%%%%%%%%%%%%%%%%%%%%%%%%%%
%
% Copyright (c) 2010 Jim Finnis.
% 
% All Rights Reserved.
% 
% This  document  may  not, in  whole  or in  part, be  copied,  photocopied,
% reproduced,  translated,  or  reduced to any  electronic  medium or machine
% readable form without prior written consent from Jim Finnis.
%
%%%%%%%%%%%%%%%%%%%%%%%%%%%%%%%%%%%%%%%%%%%%%%%%%%%%%%%%%%%%%%%%%%%%%%%%%%%%%

\chapter{Handling 404s}
\label{404chap}
If the user enters a request for a page which doesn't exist, a default 404 page will be returned. This is 
very brief, and rather ugly.

You can customise this by providing a file called \texttt{404} in the
\texttt{pages} directory, just like any other page. Most users find it a good
idea to create a \texttt{404} template as well, so that the 404 page has its
own style.

In either case, whether you create a 404 page or use the standard one, the HTTP headers will be set correctly
by SCMS to indicate the 404 status.

\section{Example}
Here's an minimal example of a 404 page, which would go into \texttt{site/pages/404}:
\begin{MyVerbatim}
name=Page Not Found
template=404

message=[[
<p>
We're sorry, but the page you requested -
<b>{{thispage}}</b> - does not exist.
<p>The home page can be found <a href="{{defaultpage}}"
    accesskey=1>here</a>.
</p>
]]
\end{MyVerbatim}
Note that the template is set to \texttt{404}, which we'll need to provide.

\clearpage
\noindent And here's a template file, which would go into \texttt{site/templates/404/main.html}:
\begin{MyVerbatim}
<html>
<title>{{page:name}}</title>
<link rel="start" title="Home Page, shortcut key=1"
    href="{{defaultpage}}">
</head>

<body>
<h1>{{page:name}}</h1>
{{page:message}}
</body></html>
\end{MyVerbatim}

Note that the \texttt{page:name} tag still has the name of the page which was requested, not that of the 404 page
which was returned.
